\documentclass[a4paper]{article}

\usepackage{relalg}
\usepackage{newtxtext}
\usepackage{newtxmath}
\usepackage[margin=1in]{geometry}

\usepackage{tikz}
\usetikzlibrary{positioning}
\usetikzlibrary{graphs}
\usetikzlibrary{decorations.pathreplacing}
\usetikzlibrary{calc}
\usetikzlibrary{arrows}

% "empty" arrow tip
\pgfarrowsdeclare{:}{:}{}{}
% custom bar arrow tip, offset from end of line (use "empty" tip at line ends if no >)
\tikzset{crossbar/.tip={|[scale width=1.75,sep=0.25em]}}
% various edge styles for TikZ
\tikzset{
    function/.style={arrows={->}},
    injection/.style={arrows={<-}},
    total/.style={arrows={:{crossbar}-}},
    surjection/.style={arrows={-{crossbar}:}},
    bijection/.style={arrows={<{crossbar}-{crossbar}>}},
    projection/.style={arrows={:{crossbar}-{crossbar}>}},
    projection left/.style={arrows={:{crossbar}-{crossbar}>},edge label={\scriptsize\(\RelProject\)}},
    projection right/.style={arrows={:{crossbar}-{crossbar}>},edge label'={\scriptsize\(\RelProject\)}},
    selection left/.style={arrows={<-{crossbar}>},edge label={\scriptsize\(\RelRestrict\)}},
    selection right/.style={arrows={<-{crossbar}>},edge label'={\scriptsize\(\RelRestrict\)}},
    funcdep left/.style={arrows={->},edge node={node[sloped,midway,above] {\scriptsize\emph{key}}}},
    funcdep right/.style={arrows={->},edge node={node[sloped,midway,below] {\scriptsize\emph{key}}}},
    surtotal/.style={arrows={:{crossbar}-{crossbar}:}},
    input keep/.style={blue,thick},
    input delete/.style={blue!40,thick,dashed},
    output/.style={red,thick},
    output temp/.style={red,thick,dashed},
    path 1/.style={green!60!black,thick},
    path 2/.style={orange,thick}
}
        
% projection and selection edge annotations for TikZ
\newcommand{\ProjectionAnnotation}[3][]{%
    \path (#2) to node[above,#1] {\scriptsize\(\RelProject\)} (#3);%
}
\newcommand{\SelectionAnnotation}[3][]{%
    \path (#2) to node[above,#1] {\scriptsize\(\RelRestrict\)} (#3);%
}


\newcounter{constraint}

% misc
\newcommand{\todo}[1]{\textbf{!!TODO!!} {[#1]}}

% commonly used elements
\newcommand{\LS}{\ensuremath{\mathit{LS}}}
\newcommand{\NLS}{\ensuremath{\mathit{NLS}}}
\newcommand{\LSsub}{\ensuremath{\mathit{L}}}
\newcommand{\NLSsub}{\ensuremath{\mathit{N}}}
\newcommand{\Sno}{\ensuremath{\mathit{Sno}}}
\newcommand{\Sname}{\ensuremath{\mathit{Sname}}}
\newcommand{\Status}{\ensuremath{\mathit{Status}}}
\newcommand{\City}{\ensuremath{\mathit{City}}}

\newcommand{\T}[1]{\ensuremath{T_{#1}}}
\newcommand{\TT}[1]{\ensuremath{T_{\{#1\}}}}

\newcommand{\CityLondon}{\ensuremath{\{\City\colon\allowbreak\mathit{'London'}\}}}
\newcommand{\TCityMinusLondon}{\ensuremath{\T{\City} \setminus \CityLondon}}
\newcommand{\TSSC}{\ensuremath{\T{\Sname} \times \T{\Status} \times \T{\City}}}
\newcommand{\TSSL}{\ensuremath{\T{\Sname} \times \T{\Status} \times \CityLondon}}
\newcommand{\TSSNL}{\ensuremath{\T{\Sname} \times \T{\Status} \times (\TCityMinusLondon)}}

\newcommand{\TLSPlusNLS}{\ensuremath{\T{\LS} + \T{\NLS}}}
\newcommand{\TTLSPlusNLS}{\ensuremath{\TT{\LS} + \TT{\NLS}}}
\newcommand{\TLSPlusNLSsub}{\ensuremath{\T{\LSsub} + \T{\NLSsub}}}
\newcommand{\TTLSPlusNLSsub}{\ensuremath{\TT{\LSsub} + \TT{\NLSsub}}}

\newcommand{\StackTLSPlusNLS}{\ensuremath{\begin{array}{c}\T{\LS}\,+ \\ \T{\NLS}\end{array}}}
\newcommand{\StackTTLSPlusNLS}{\ensuremath{\begin{array}{c}\TT{\LS}\,+ \\ \TT{\NLS}\end{array}}}
\newcommand{\StackTLSPlusNLSsub}{\ensuremath{\begin{array}{c}\T{\LSsub}\,+ \\ \T{\NLSsub}\end{array}}}
\newcommand{\StackTTLSPlusNLSsub}{\ensuremath{\begin{array}{c}\TT{\LSsub}\,+ \\ \TT{\NLSsub}\end{array}}}
\newcommand{\StackTSSC}{\ensuremath{\begin{array}{c}\T{\Sname}\,\times \\ \T{\Status} \times \T{\City}\end{array}}}
\newcommand{\StackTSSL}{\ensuremath{\begin{array}{c}\T{\Sname} \times \T{\Status}\,\times \\ \CityLondon\end{array}}}
\newcommand{\StackTSSNL}{\ensuremath{\begin{array}{c}\T{\Sname} \times \T{\Status}\,\times \\ (\TCityMinusLondon)\end{array}}}
\newcommand{\StackTCityMinusLondon}{\ensuremath{\begin{array}{c}\T{\City}\,\setminus \\ \CityLondon\end{array}}}

\newcommand{\SC}[1]{\ensuremath{\mathcal{S}_{#1}}}

% dominates
\newcommand{\Dominates}[2]{\ensuremath{#2 \preceq #1}}
\newcommand{\Equivalent}[2]{\ensuremath{#1 \equiv #2}}

% SIG notation (in text)
\newcommand{\Sedge}[1]{\ensuremath{\sigma_{\textrm{#1}}}}
\newcommand{\SedgeP}[1]{\ensuremath{\sigma_{\textrm{#1}}^{'}}}

\newcommand{\medmid}{\raise.125ex\hbox{\scalebox{1}[0.75]{$\mid$}}}

% General SIG edges for use in formulas.
% Adapted from: http://tex.stackexchange.com/questions/96330/adding-symbols-at-the-ends-of-a-horizontal-line
\makeatletter
\newlength{\@annotskipleft}
\newlength{\@annotskipright}
% #1 = left edge component
% #2 = right edge component
% #3 = left bar annotation
% #4 = right bar annotation
\newcommand\@sig@edge[4]{%
    \let\@middle\joinrel%
    \ifx#1\relbar%
        \@annotskipleft=.3em%
        % scrunch up the bare line so it's similar length to \long...arrow
        \ifx#2\relbar\def\@middle{\joinrel\joinrel\relbar\joinrel\joinrel}\fi%
    \else% 
        % arrows need a little more clearance
        \@annotskipleft=.4em%
    \fi%
    \ifx#2\relbar\@annotskipright=.3em\else\@annotskipright=.4em\fi%
    \mathrel{\ooalign{$#1\@middle#2$\cr\hskip\@annotskipleft$#3$\hfil$#4$\hskip\@annotskipright\cr}}%
}

% 0 = nothing
% 1 = bar
% 2 = arrowhead
% 3 = both
\def\@sigedge#1#2{%
    \ifcase\numexpr#1*4+#2\relax%
        \@sig@edge{\relbar}{\relbar}{}{}\or                     % 00 = -----
        \@sig@edge{\relbar}{\relbar}{}{\medmid}\or              % 01 = ---+-
        \@sig@edge{\relbar}{\rightarrow}{}{}\or                 % 02 = ---->
        \@sig@edge{\relbar}{\rightarrow}{}{\medmid}\or          % 03 = ---+>
        \@sig@edge{\relbar}{\relbar}{\medmid}{}\or              % 10 = -+---
        \@sig@edge{\relbar}{\relbar}{\medmid}{\medmid}\or       % 11 = -+-+-
        \@sig@edge{\relbar}{\rightarrow}{\medmid}{}\or          % 12 = -+-->
        \@sig@edge{\relbar}{\rightarrow}{\medmid}{\medmid}\or   % 13 = -+-+>
        \@sig@edge{\leftarrow}{\relbar}{}{}\or                  % 20 = <----
        \@sig@edge{\leftarrow}{\relbar}{}{\medmid}\or           % 21 = <--+-
        \@sig@edge{\leftarrow}{\rightarrow}{}{}\or              % 22 = <--->
        \@sig@edge{\leftarrow}{\rightarrow}{}{\medmid}\or       % 23 = <--+>
        \@sig@edge{\leftarrow}{\relbar}{\medmid}{}\or           % 30 = <+---
        \@sig@edge{\leftarrow}{\relbar}{\medmid}{\medmid}\or    % 31 = <+-+-
        \@sig@edge{\leftarrow}{\rightarrow}{\medmid}{}\or       % 32 = <+-->
        \@sig@edge{\leftarrow}{\rightarrow}{\medmid}{\medmid}   % 33 = <+-+>
    \fi%
}

\newcommand{\sigedge}[1]{\ensuremath{\@sigedge#1}}
\makeatother

\newcommand{\Edge}{\sigedge{00}}
\newcommand{\Total}{\sigedge{10}}
\newcommand{\Surjective}{\sigedge{01}}
\newcommand{\SurTotal}{\sigedge{11}}
\newcommand{\Functional}{\sigedge{02}}
\newcommand{\Injective}{\sigedge{20}}
\newcommand{\Bijective}{\sigedge{33}}

% Hack to get the overset symbols to appear at the right height.
% \smash removes the spacing around the operator, hence \mathop.
\newcommand{\LabelledEdge}[2]{\mathop{\overset{\raisebox{0.3em}{\scriptsize\ensuremath{#2}}}{\smash[t]{#1}}}}
\newcommand{\ProjectionEdge}{\LabelledEdge{\sigedge{13}}{\RelProject}}
\newcommand{\SelectionEdge}{\LabelledEdge{\sigedge{23}}{\RelRestrict}}
\newcommand{\TrivialProjection}{\ensuremath{\LabelledEdge{\Bijective}{\RelProject}}}
\newcommand{\TrivialSelection}{\ensuremath{\LabelledEdge{\Bijective}{\RelRestrict}}}
\newcommand{\KeyEdge}{\ensuremath{\LabelledEdge{\Functional}{\mathit{key}}}}

% Constraints.
\newcommand{\Constraint}[2][]{C\ensuremath{_{#2}\ifx&#1&\else^{#1}\fi}}
\newenvironment{ConstraintList}[1][]{%
    \begin{list}{%
        \bfseries%
        \ifx&#1&%
            \Constraint{\ensuremath{\bm{\arabic{constraint}}}}%
        \else%
            \Constraint[\ensuremath{\bm{#1}}]{\ensuremath{\bm{\arabic{constraint}}}}%
        \fi%
    }%
    {\usecounter{constraint}}%
}{\end{list}}


\hyphenation{co-pied}


%%%%%%%%%%%%%%%%%%%%%%%%%%%%%%%%%%%%%%%%%%%%%%%%%%%%%%%%%%%%%%%%%%%%%%%%%%%%%%%%%


\begin{document}

%%%%%%%%%%%%%%%%%%%%

\begin{figure*}
    \centering
    \begin{tikzpicture}
        \matrix[row sep=0.5cm]
        { 
            \node (TLS) {\(\T{\LSsub}\)};   &[7mm]  \node (TTLS) {\(\TT{\LSsub}\)};     &[7mm]                              &[4mm]                                  &[-4mm] \node (TSSL) {\(\StackTSSL\)};      &                                       &   \node (LSCity) {\(\CityLondon\)};   \\
                                            &                                           &                                   &   \node (TSname) {\(\T{\Sname}\)};    \\
            \node (TLSPlusNLS) {\(\T{S}\)}; &       \node (TTLSPlusNLS) {\(\TT{S}\)};   &   \node (TSno) {\(\T{\Sno}\)};    &                                       &       \node (TSSC) {\(\StackTSSC\)};      &                                       &   \node (TCity) {\(\T{City}\)};   \\
                                            &                                           &                                   &                                       &                                           &   \node (TStatus) {\(\T{\Status}\)};  \\
            \node (TNLS) {\(\T{\NLSsub}\)}; &       \node (TTNLS) {\(\TT{\NLSsub}\)};   &                                   &                                       &       \node (TSSNL) {\(\StackTSSNL\)};    &                                       &   \node (NLSCity) {\(\StackTCityMinusLondon\)}; \\
        };
        
        \graph{
            % relation types to tuple types
            { [edges=surtotal]
                { (TLS), (TNLS), (TLSPlusNLS) } -- { (TTLS), (TTNLS), (TTLSPlusNLS) },
            };
            
            % FDs
            (TSno) -- [funcdep left,bend left] (TSSL);
            (TSno) -- [funcdep right] (TSSNL);
            
            % projection edges
            { [edges=projection left]
                (TTLSPlusNLS) -- (TSno),
                (TTLS)  -- { (TSno), (TSSL) },
                (TSSL)  -- { (TStatus), (LSCity) },
            };
            
            { [edges=projection right]
                (TTNLS) -- { (TSno), (TSSNL) },
                (TSSC)  -- { (TSname),  (TStatus) },
                (TSSL)  -- (TSname),
                (TSSNL) -- { (TStatus), (NLSCity), (TSname) },
            };
            
            % selection edges
            { [edges=selection left]
                { (TLSPlusNLS), (TTLSPlusNLS) } -- { (TLS), (TTLS) },
                (TSSC)  -- { (TSSL), (TSSNL) },
                (TCity) -- (NLSCity),
            };
            { [edges=selection right]
                { (TLSPlusNLS), (TTLSPlusNLS) } -- { (TNLS), (TTNLS) },
                (TCity) -- (LSCity),
            };
            
            % insert "gaps" into crossed edges
            { [edges={white,line width=1.5mm}]
                (TSno) -- (TSSC),
                (TSSC) -- (TCity),
                (TTLSPlusNLS) -- [bend right] (TSSC),
            };
            (TSno) -- [funcdep right] (TSSC);
            (TSSC) -- [projection left] (TCity);
            (TTLSPlusNLS) -- [projection right,bend right] (TSSC),
        };
    \end{tikzpicture}
    \caption{SIG for schema \(\SC{1} = \{S\}\).}
    \label{fig-sig-s}
\end{figure*}

%%%%%%%%%%%%%%%%%%%%

%%%%%%%%%%%%%%%%%%%%

\begin{figure*}
    \centering
    \begin{tikzpicture}
        \matrix[row sep=0.5cm]
        { 
            \node (TLS) {\(\T{\LS}\)};                  &[7mm]  \node (TTLS) {\(\TT{\LS}\)};                    &[7mm]                              &[4mm]                                  &[-4mm] \node (TSSL) {\(\StackTSSL\)};      &                                       &   \node (LSCity) {\(\CityLondon\)};   \\
                                                        &                                                       &                                   &   \node (TSname) {\(\T{\Sname}\)};    \\
            \node (TLSPlusNLS) {\(\StackTLSPlusNLS\)};  &       \node (TTLSPlusNLS) {\(\StackTTLSPlusNLS\)};    &   \node (TSno) {\(\T{\Sno}\)};    &                                       &       \node (TSSC) {\(\StackTSSC\)};      &                                       &   \node (TCity) {\(\T{City}\)};   \\
                                                        &                                                       &                                   &                                       &                                           &   \node (TStatus) {\(\T{\Status}\)};  \\
            \node (TNLS) {\(\T{\NLS}\)};                &       \node (TTNLS) {\(\TT{\NLS}\)};                  &                                   &                                       &       \node (TSSNL) {\(\StackTSSNL\)};    &                                       &   \node (NLSCity) {\(\StackTCityMinusLondon\)}; \\
        };
        
        \graph{
            % relation types to tuple types
            { [edges=surtotal]
                { (TLS), (TNLS), (TLSPlusNLS) } -- { (TTLS), (TTNLS), (TTLSPlusNLS) },
            };
            
            % FDs
            (TSno) -- [funcdep left,bend left] (TSSL);
            (TSno) -- [funcdep right] (TSSNL);
            
            % projection edges
            { [edges=projection left]
                (TTLSPlusNLS) -- (TSno),
                (TTLS)  -- { (TSno), (TSSL) },
                (TSSL)  -- { (TStatus), (LSCity) },
            };
            
            { [edges=projection right]
                (TTNLS) -- { (TSno), (TSSNL) },
                (TSSC)  -- { (TSname),  (TStatus) },
                (TSSL)  -- (TSname),
                (TSSNL) -- { (TStatus), (NLSCity), (TSname) },
            };
            
            % selection edges
            { [edges=selection left]
                { (TLSPlusNLS), (TTLSPlusNLS) } -- { (TLS), (TTLS) },
                (TSSC)  -- { (TSSL), (TSSNL) },
                (TCity) -- (NLSCity),
            };
            { [edges=selection right]
                { (TLSPlusNLS), (TTLSPlusNLS) } -- { (TNLS), (TTNLS) },
                (TCity) -- (LSCity),
            };
            
            % insert "gaps" into crossed edges
            { [edges={white,line width=1.5mm}]
                (TSno) -- (TSSC),
                (TSSC) -- (TCity),
                (TTLSPlusNLS) -- [bend right] (TSSC),
            };
            (TSno) -- [funcdep right] (TSSC);
            (TSSC) -- [projection left] (TCity);
            (TTLSPlusNLS) -- [projection right,bend right] (TSSC),
        };
    \end{tikzpicture}
    \caption{SIG for schema \(\SC{2} = \{\LS, \NLS\}\).}
    \label{fig-sig-ls-nls}
\end{figure*}

%%%%%%%%%%%%%%%%%%%%

%%%%%%%%%%%%%%%%%%%%

\begin{figure*}
    \centering
    \begin{tikzpicture}
        \matrix[row sep=0.5cm]
        { 
            \node (TLS) {\(\T{\LS}\)};  &[7.5mm]  \node (TTLS) {\(\TT{\LS}\)};  &[7mm]                              &[4mm]                                  &[-4mm] \node (TSSL) {\(\StackTSSL\)};      &                                       &   \node (LSCity) {\(\CityLondon\)};   \\
                                        &                                       &                                   &   \node (TSname) {\(\T{\Sname}\)};    \\
                                        &                                       &   \node (TSno) {\(\T{\Sno}\)};    &                                       &       \node (TSSC) {\(\StackTSSC\)};      &                                       &   \node (TCity) {\(\T{City}\)};   \\
                                        &                                       &                                   &                                       &                                           &   \node (TStatus) {\(\T{\Status}\)};  \\
                                        &                                       &                                   &                                       &       \node (TSSNL) {\(\StackTSSNL\)};    &                                       &   \node (NLSCity) {\(\StackTCityMinusLondon\)}; \\
        };
        
        \graph{
            % relation types to tuple types
            { [edges=surtotal]
                (TLS) -- (TTLS),
            };
            
            % projection edges
            { [edges=projection left]
                (TTLS)  -- { (TSno), (TSSL) },
                (TSSL)  -- { (TStatus), (LSCity) },
            };
            
            { [edges=projection right]
                (TSSC)  -- { (TSname),  (TStatus) },
                (TSSL)  -- (TSname),
                (TSSNL) -- { (TStatus), (NLSCity), (TSname) },
            };
            
            % selection edges
            { [edges=selection left]
                (TSSC)  -- { (TSSL), (TSSNL) },
                (TCity) -- (NLSCity),
            };
            { [edges=selection right]
                (TCity) -- (LSCity),
            };
            
            % insert "gaps" into crossed edges
            { [edges={white,line width=1.5mm}]
                (TSno) -- (TSSC),
                (TSSC) -- (TCity),
            };
            (TSno) -- [funcdep right] (TSSC);
            (TSSC) -- [projection left] (TCity);
        };
    \end{tikzpicture}
    \caption{SIG for schema \(\SC{3} = \{\LS\}\).}
    \label{fig-sig-ls}
\end{figure*}

%%%%%%%%%%%%%%%%%%%%



%%%%%%%%%%%%%%%%%%%%

\begin{figure*}
    \centering
    \begin{tikzpicture}
        \matrix[row sep=0.5cm]
        { 
            \node [input keep] (TLS) {\(\T{\LS}\)};     &[7mm]  \node [input keep] (TTLS) {\(\TT{\LS}\)};       &[7mm]                              &[4mm]                                  &[-4mm] \node (TSSL) {\(\StackTSSL\)};      &                                       &   \node (LSCity) {\(\CityLondon\)};   \\
                                                        &                                                       &                                   &   \node (TSname) {\(\T{\Sname}\)};    \\
            \node [output] (TLSPlusNLS) {\(\T{S}\)};    &       \node [output] (TTLSPlusNLS) {\(\TT{S}\)};      &   \node (TSno) {\(\T{\Sno}\)};    &                                       &       \node (TSSC) {\(\StackTSSC\)};      &                                       &   \node (TCity) {\(\T{City}\)};   \\
                                                        &                                                       &                                   &                                       &                                           &   \node (TStatus) {\(\T{\Status}\)};  \\
            \node [transparent] (TNLS) {\(\T{\NLS}\)};  &       \node [transparent] (TTNLS) {\(\TT{\NLS}\)};    &                                   &                                       &       \node (TSSNL) {\(\StackTSSNL\)};    &                                       &   \node (NLSCity) {\(\StackTCityMinusLondon\)}; \\
        };
        
        \graph{
            % relation types to tuple types
            { [edges=surtotal]
                (TLS)        -- (TTLS),
                (TLSPlusNLS) -- [transparent] (TTLSPlusNLS),
                (TNLS)       -- [transparent] (TTNLS),
            };
            
            % FDs
            (TSno) -- [funcdep left,bend left,transparent] (TSSL);
            (TSno) -- [funcdep right,transparent] (TSSNL);
            
            % projection edges
            { [edges=projection left]
                (TTLS)  -- { (TSno), (TSSL) },
                (TSSL)  -- { (TStatus), (LSCity) },
                (TTLSPlusNLS) -- [transparent] (TSno),
            };
            
            { [edges=projection right]
                (TSSC)  -- { (TSname),  (TStatus) },
                (TSSL)  -- (TSname),
                (TSSNL) -- { (TStatus), (NLSCity), (TSname) },
                (TTNLS)  -- [transparent] (TSno),
                (TTNLS)  -- [arrows={:->},transparent] (TSSNL),
            };
            
            % selection edges
            { [edges=selection left]
                (TSSC)  -- { (TSSL), (TSSNL) },
                (TCity) -- (NLSCity),
            };
            { [edges=selection right]
                (TCity) -- (LSCity),
            };
            
            { [edges=bijection,edge label={\scriptsize\(\RelRestrict\)}]
                { (TLS), (TTLS) }               -- [output]      { (TLSPlusNLS), (TTLSPlusNLS) },
                { (TLSPlusNLS), (TTLSPlusNLS) } -- [transparent] { (TNLS), (TTNLS) },
            };
            
            % insert "gaps" into crossed edges
            { [edges={white,line width=1.5mm}]
                (TSno) -- (TSSC),
                (TSSC) -- (TCity),
                (TTLSPlusNLS) -- [bend right] (TSSC),
            };
            (TSno) -- [funcdep right] (TSSC);
            (TSSC) -- [projection left] (TCity);
            (TTLSPlusNLS) -- [projection right,arrows={:{crossbar}->},bend right,transparent] (TSSC),
        };
    \end{tikzpicture}
    \caption{SIG for transformed schema \(S_{3}'\): duplicate \(\T{\LS}\) to \(\T{S}\) and \(\TT{\LS}\) to \(\TT{S}\). Inputs to the transformations are coloured \textcolor{blue}{blue}, outputs are coloured \textcolor{red}{red}. [\(\equiv\)]}
%     \label{fig-transform-edge-moves}
\end{figure*}

%%%%%%%%%%%%%%%%%%%%

%%%%%%%%%%%%%%%%%%%%

\begin{figure*}
    \centering
    \begin{tikzpicture}
        \matrix[row sep=0.5cm]
        { 
            \node (TLS) {\(\T{\LS}\)};                      &[7mm]  \node (TTLS) {\(\TT{\LS}\)};                    &[7mm]                              &[4mm]                                  &[-4mm] \node (TSSL) {\(\StackTSSL\)};      &                                       &   \node (LSCity) {\(\CityLondon\)};   \\
                                                            &                                                       &                                   &   \node (TSname) {\(\T{\Sname}\)};    \\
            \node [input keep] (TLSPlusNLS) {\(\T{S}\)};    &       \node [input keep] (TTLSPlusNLS) {\(\TT{S}\)};  &   \node (TSno) {\(\T{\Sno}\)};    &                                       &       \node (TSSC) {\(\StackTSSC\)};      &                                       &   \node (TCity) {\(\T{City}\)};   \\
                                                            &                                                       &                                   &                                       &                                           &   \node (TStatus) {\(\T{\Status}\)};  \\
            \node [output] (TNLS) {\(\T{\NLS}\)};           &       \node [output] (TTNLS) {\(\TT{\NLS}\)};         &                                   &                                       &       \node (TSSNL) {\(\StackTSSNL\)};    &                                       &   \node (NLSCity) {\(\StackTCityMinusLondon\)}; \\
        };
        
        \graph{
            % relation types to tuple types
            { [edges=surtotal]
                (TLS)        -- (TTLS),
                (TLSPlusNLS) -- [transparent] (TTLSPlusNLS),
                (TNLS)       -- [transparent] (TTNLS),
            };
            
            % FDs
            (TSno) -- [funcdep left,bend left,transparent] (TSSL);
            (TSno) -- [funcdep right,transparent] (TSSNL);
            
            % projection edges
            { [edges=projection left]
                (TTLS)  -- { (TSno), (TSSL) },
                (TSSL)  -- { (TStatus), (LSCity) },
                (TTLSPlusNLS) -- [transparent] (TSno),
            };
            
            { [edges=projection right]
                (TSSC)  -- { (TSname),  (TStatus) },
                (TSSL)  -- (TSname),
                (TSSNL) -- { (TStatus), (NLSCity), (TSname) },
                (TTNLS)  -- [transparent] (TSno),
                (TTNLS)  -- [arrows={:->},transparent] (TSSNL),
            };
            
            % selection edges
            { [edges=selection left]
                (TSSC)  -- { (TSSL), (TSSNL) },
                (TCity) -- (NLSCity),
            };
            { [edges=selection right]
                (TCity) -- (LSCity),
            };
            
            { [edges=bijection,edge label={\scriptsize\(\RelRestrict\)}]
                { (TLS), (TTLS) }               --          { (TLSPlusNLS), (TTLSPlusNLS) },
                { (TLSPlusNLS), (TTLSPlusNLS) } -- [output] { (TNLS), (TTNLS) },
            };
            
            % insert "gaps" into crossed edges
            { [edges={white,line width=1.5mm}]
                (TSno) -- (TSSC),
                (TSSC) -- (TCity),
                (TTLSPlusNLS) -- [bend right] (TSSC),
            };
            (TSno) -- [funcdep right] (TSSC);
            (TSSC) -- [projection left] (TCity);
            (TTLSPlusNLS) -- [projection right,arrows={:{crossbar}->},bend right,transparent] (TSSC),
        };
    \end{tikzpicture}
    \caption{SIG for transformed schema \(S_{3}'\): duplicate \(\T{S}\) to \(\T{\NLS}\) and \(\TT{S}\) to \(\TT{\NLS}\). [\(\equiv\)]}
%     \label{fig-transform-edge-moves}
\end{figure*}

%%%%%%%%%%%%%%%%%%%%

%%%%%%%%%%%%%%%%%%%%

\begin{figure*}
    \centering
    \begin{tikzpicture}
        \matrix[row sep=0.5cm]
        { 
            \node (TLS) {\(\T{\LS}\)};      &[7mm]  \node (TTLS) {\(\TT{\LS}\)};        &[7mm]                              &[4mm]                                  &[-4mm] \node (TSSL) {\(\StackTSSL\)};      &                                       &   \node (LSCity) {\(\CityLondon\)};   \\
                                            &                                           &                                   &   \node (TSname) {\(\T{\Sname}\)};    \\
            \node (TLSPlusNLS) {\(\T{S}\)}; &       \node (TTLSPlusNLS) {\(\TT{S}\)};   &   \node (TSno) {\(\T{\Sno}\)};    &                                       &       \node (TSSC) {\(\StackTSSC\)};      &                                       &   \node (TCity) {\(\T{City}\)};   \\
                                            &                                           &                                   &                                       &                                           &   \node (TStatus) {\(\T{\Status}\)};  \\
            \node (TNLS) {\(\T{\NLS}\)};    &       \node (TTNLS) {\(\TT{\NLS}\)};      &                                   &                                       &       \node (TSSNL) {\(\StackTSSNL\)};    &                                       &   \node (NLSCity) {\(\StackTCityMinusLondon\)}; \\
        };
        
        \graph{
            % relation types to tuple types
            { [edges=surtotal]
                (TLS)        -- [input keep] (TTLS),
                (TLSPlusNLS) -- [output]     (TTLSPlusNLS),
                (TNLS)       -- [output]     (TTNLS),
            };
            
            % FDs
            (TSno) -- [funcdep left,bend left,transparent] (TSSL);
            (TSno) -- [funcdep right,transparent] (TSSNL);
            
            % projection edges
            { [edges=projection left]
                (TTLS)  -- { (TSno), (TSSL) },
                (TSSL)  -- { (TStatus), (LSCity) },
                (TTLSPlusNLS) -- [transparent] (TSno),
            };
            
            { [edges=projection right]
                (TSSC)  -- { (TSname),  (TStatus) },
                (TSSL)  -- (TSname),
                (TSSNL) -- { (TStatus), (NLSCity), (TSname) },
                (TTNLS)  -- [transparent] (TSno),
                (TTNLS)  -- [arrows={:->},transparent] (TSSNL),
            };
            
            % selection edges
            { [edges=selection left]
                (TSSC)  -- { (TSSL), (TSSNL) },
                (TCity) -- (NLSCity),
            };
            { [edges=selection right]
                (TCity) -- (LSCity),
            };
            
            { [edges=bijection,edge label={\scriptsize\(\RelRestrict\)}]
                { (TLS), (TTLS) }               -- { (TLSPlusNLS), (TTLSPlusNLS) },
                { (TLSPlusNLS), (TTLSPlusNLS) } -- { (TNLS), (TTNLS) },
            };
            
            % insert "gaps" into crossed edges
            { [edges={white,line width=1.5mm}]
                (TSno) -- (TSSC),
                (TSSC) -- (TCity),
                (TTLSPlusNLS) -- [bend right] (TSSC),
            };
            (TSno) -- [funcdep right] (TSSC);
            (TSSC) -- [projection left] (TCity);
            (TTLSPlusNLS) -- [projection right,arrows={:{crossbar}->},bend right,transparent] (TSSC),
        };
    \end{tikzpicture}
    \caption{SIG for transformed schema \(S_{3}'\): copy edge \(\T{\LS} \protect\SurTotal \TT{\LS}\) to \(\T{S} \protect\SurTotal \TT{S}\) and \(\T{\NLS} \protect\SurTotal \TT{\NLS}\). [\(\preceq\)]}
%     \label{fig-transform-edge-moves}
\end{figure*}

%%%%%%%%%%%%%%%%%%%%

%%%%%%%%%%%%%%%%%%%%

\begin{figure*}
    \centering
    \begin{tikzpicture}
        \matrix[row sep=0.5cm]
        { 
            \node (TLS) {\(\T{\LS}\)};      &[7mm]  \node (TTLS) {\(\TT{\LS}\)};        &[7mm]                              &[4mm]                                  &[-4mm] \node (TSSL) {\(\StackTSSL\)};      &                                       &   \node (LSCity) {\(\CityLondon\)};   \\
                                            &                                           &                                   &   \node (TSname) {\(\T{\Sname}\)};    \\
            \node (TLSPlusNLS) {\(\T{S}\)}; &       \node (TTLSPlusNLS) {\(\TT{S}\)};   &   \node (TSno) {\(\T{\Sno}\)};    &                                       &       \node (TSSC) {\(\StackTSSC\)};      &                                       &   \node (TCity) {\(\T{City}\)};   \\
                                            &                                           &                                   &                                       &                                           &   \node (TStatus) {\(\T{\Status}\)};  \\
            \node (TNLS) {\(\T{\NLS}\)};    &       \node (TTNLS) {\(\TT{\NLS}\)};      &                                   &                                       &       \node (TSSNL) {\(\StackTSSNL\)};    &                                       &   \node (NLSCity) {\(\StackTCityMinusLondon\)}; \\
        };
        
        \graph{
            % relation types to tuple types
            { [edges=surtotal]
                (TLS)        -- (TTLS),
                (TLSPlusNLS) -- (TTLSPlusNLS),
                (TNLS)       -- (TTNLS),
            };
            
            % FDs
            (TSno) -- [funcdep left,bend left,transparent] (TSSL);
            (TSno) -- [funcdep right,transparent] (TSSNL);
            
            % projection edges
            { [edges=projection left]
                (TTLS)  -- [input keep] (TSno),
                (TTLS)  -- (TSSL),
                (TSSL)  -- { (TStatus), (LSCity) },
                (TTLSPlusNLS) -- [output] (TSno),
            };
            
            { [edges=projection right]
                (TSSC)  -- { (TSname),  (TStatus) },
                (TSSL)  -- (TSname),
                (TSSNL) -- { (TStatus), (NLSCity), (TSname) },
                (TTNLS)  -- [output] (TSno),
                (TTNLS)  -- [arrows={:->},transparent] (TSSNL),
            };
            
            % selection edges
            { [edges=selection left]
                (TSSC)  -- { (TSSL), (TSSNL) },
                (TCity) -- (NLSCity),
            };
            { [edges=selection right]
                (TCity) -- (LSCity),
            };
            
            { [edges=bijection,edge label={\scriptsize\(\RelRestrict\)}]
                { (TLS), (TTLS) }               -- { (TLSPlusNLS), (TTLSPlusNLS) },
                { (TLSPlusNLS), (TTLSPlusNLS) } -- { (TNLS), (TTNLS) },
            };
            
            % insert "gaps" into crossed edges
            { [edges={white,line width=1.5mm}]
                (TSno) -- (TSSC),
                (TSSC) -- (TCity),
                (TTLSPlusNLS) -- [bend right,transparent] (TSSC),
            };
            (TSno) -- [funcdep right] (TSSC);
            (TSSC) -- [projection left] (TCity);
            (TTLSPlusNLS) -- [projection right,arrows={:{crossbar}->},bend right,transparent] (TSSC),
        };
    \end{tikzpicture}
    \caption{SIG for transformed schema \(S_{3}'\): copy edge \(\TT{\LS} \protect\ProjectionEdge \T{\Sno}\) to \(\TT{S} \protect\ProjectionEdge \T{\Sno}\) and \(\TT{\NLS} \protect\ProjectionEdge \T{\Sno}\). [\(\preceq\)]}
%     \label{fig-transform-edge-moves}
\end{figure*}

%%%%%%%%%%%%%%%%%%%%

%%%%%%%%%%%%%%%%%%%%

\begin{figure*}
    \centering
    \begin{tikzpicture}
        \matrix[row sep=0.5cm]
        { 
            \node (TLS) {\(\T{\LS}\)};      &[7mm]  \node (TTLS) {\(\TT{\LS}\)};        &[7mm]                              &[4mm]                                  &[-4mm] \node (TSSL) {\(\StackTSSL\)};      &                                       &   \node (LSCity) {\(\CityLondon\)};   \\
                                            &                                           &                                   &   \node (TSname) {\(\T{\Sname}\)};    \\
            \node (TLSPlusNLS) {\(\T{S}\)}; &       \node (TTLSPlusNLS) {\(\TT{S}\)};   &   \node (TSno) {\(\T{\Sno}\)};    &                                       &       \node (TSSC) {\(\StackTSSC\)};      &                                       &   \node (TCity) {\(\T{City}\)};   \\
                                            &                                           &                                   &                                       &                                           &   \node (TStatus) {\(\T{\Status}\)};  \\
            \node (TNLS) {\(\T{\NLS}\)};    &       \node (TTNLS) {\(\TT{\NLS}\)};      &                                   &                                       &       \node (TSSNL) {\(\StackTSSNL\)};    &                                       &   \node (NLSCity) {\(\StackTCityMinusLondon\)}; \\
        };
        
        \graph{
            % relation types to tuple types
            { [edges=surtotal]
                (TLS)        -- (TTLS),
                (TLSPlusNLS) -- (TTLSPlusNLS),
                (TNLS)       -- (TTNLS),
            };
            
            % FDs
            (TSno) -- [funcdep left,bend left,transparent] (TSSL);
            (TSno) -- [funcdep right,transparent] (TSSNL);
            
            % projection edges
            { [edges=projection left]
                (TTLS)  -- (TSno),
                (TTLS)  -- [input keep] (TSSL),
                (TSSL)  -- { (TStatus), (LSCity) },
                (TTLSPlusNLS) -- (TSno),
            };
            
            { [edges=projection right]
                (TSSC)  -- { (TSname),  (TStatus) },
                (TSSL)  -- (TSname),
                (TSSNL) -- { (TStatus), (NLSCity), (TSname) },
                (TTNLS)  -- (TSno),
                (TTNLS)  -- [arrows={:->},output] (TSSNL),
            };
            
            % selection edges
            { [edges=selection left]
                (TSSC)  -- { (TSSL), (TSSNL) },
                (TCity) -- (NLSCity),
            };
            { [edges=selection right]
                (TCity) -- (LSCity),
            };
            
            { [edges=bijection,edge label={\scriptsize\(\RelRestrict\)}]
                { (TLS), (TTLS) }               -- { (TLSPlusNLS), (TTLSPlusNLS) },
                { (TLSPlusNLS), (TTLSPlusNLS) } -- { (TNLS), (TTNLS) },
            };
            
            % insert "gaps" into crossed edges
            { [edges={white,line width=1.5mm}]
                (TSno) -- (TSSC),
                (TSSC) -- (TCity),
                (TTLSPlusNLS) -- [bend right] (TSSC),
            };
            (TSno) -- [funcdep right] (TSSC);
            (TSSC) -- [projection left] (TCity);
            (TTLSPlusNLS) -- [projection right,arrows={:{crossbar}->},bend right,output] (TSSC),
        };
    \end{tikzpicture}
    \caption{SIG for transformed schema \(S_{3}'\): copy edge \(\TT{\LS} \protect\ProjectionEdge \TSSL\) to \(\TT{S} \protect\LabelledEdge{\protect\sigedge{12}}{\RelProject} \TSSC\) and \(\TT{\NLS} \protect\LabelledEdge{\protect\sigedge{02}}{\RelProject} \TSSNL\). [\(\preceq\)]}
%     \label{fig-transform-edge-moves}
\end{figure*}

%%%%%%%%%%%%%%%%%%%%

%%%%%%%%%%%%%%%%%%%%

\begin{figure*}
    \centering
    \begin{tikzpicture}
        \matrix[row sep=0.5cm]
        { 
            \node (TLS) {\(\T{\LS}\)};      &[7mm]  \node (TTLS) {\(\TT{\LS}\)};        &[7mm]                              &[4mm]                                  &[-4mm] \node (TSSL) {\(\StackTSSL\)};      &                                       &   \node (LSCity) {\(\CityLondon\)};   \\
                                            &                                           &                                   &   \node (TSname) {\(\T{\Sname}\)};    \\
            \node (TLSPlusNLS) {\(\T{S}\)}; &       \node (TTLSPlusNLS) {\(\TT{S}\)};   &   \node (TSno) {\(\T{\Sno}\)};    &                                       &       \node (TSSC) {\(\StackTSSC\)};      &                                       &   \node (TCity) {\(\T{City}\)};   \\
                                            &                                           &                                   &                                       &                                           &   \node (TStatus) {\(\T{\Status}\)};  \\
            \node (TNLS) {\(\T{\NLS}\)};    &       \node (TTNLS) {\(\TT{\NLS}\)};      &                                   &                                       &       \node (TSSNL) {\(\StackTSSNL\)};    &                                       &   \node (NLSCity) {\(\StackTCityMinusLondon\)}; \\
        };
        
        \graph{
            % relation types to tuple types
            { [edges=surtotal]
                (TLS)        -- (TTLS),
                (TLSPlusNLS) -- (TTLSPlusNLS),
                (TNLS)       -- (TTNLS),
            };
            
            % FDs
            (TSno) -- [funcdep left,bend left,output] (TSSL);
            (TSno) -- [funcdep right,output] (TSSNL);
            
            % projection edges
            { [edges=projection left]
                (TTLS)  -- (TSno),
                (TTLS)  -- (TSSL),
                (TSSL)  -- { (TStatus), (LSCity) },
                (TTLSPlusNLS) -- (TSno),
            };
            
            { [edges=projection right]
                (TSSC)  -- { (TSname),  (TStatus) },
                (TSSL)  -- (TSname),
                (TSSNL) -- { (TStatus), (NLSCity), (TSname) },
                (TTNLS)  -- (TSno),
                (TTNLS)  -- [arrows={:->}] (TSSNL),
            };
            
            % selection edges
            { [edges=selection left]
                (TSSC)  -- { (TSSL), (TSSNL) },
                (TCity) -- (NLSCity),
            };
            { [edges=selection right]
                (TCity) -- (LSCity),
            };
            
            { [edges=bijection,edge label={\scriptsize\(\RelRestrict\)}]
                { (TLS), (TTLS) }               -- { (TLSPlusNLS), (TTLSPlusNLS) },
                { (TLSPlusNLS), (TTLSPlusNLS) } -- { (TNLS), (TTNLS) },
            };
            
            % insert "gaps" into crossed edges
            { [edges={white,line width=1.5mm}]
                (TSno) -- (TSSC),
                (TSSC) -- (TCity),
                (TTLSPlusNLS) -- [bend right] (TSSC),
            };
            (TSno) -- [funcdep right,input keep] (TSSC);
            (TSSC) -- [projection left] (TCity);
            (TTLSPlusNLS) -- [projection right,arrows={:{crossbar}->},bend right] (TSSC),
        };
    \end{tikzpicture}
    \caption{SIG for transformed schema \(S_{3}'\): copy edge \(\T{\Sno} \protect\KeyEdge \TSSC\) to \(\T{\Sno} \protect\KeyEdge \TSSL\) and \(\T{\Sno} \protect\KeyEdge \TSSNL\). [\(\preceq\)]}
%     \label{fig-transform-edge-moves}
\end{figure*}

%%%%%%%%%%%%%%%%%%%%

%%%%%%%%%%%%%%%%%%%%

\begin{figure*}
    \centering
    \begin{tikzpicture}
        \matrix[row sep=0.5cm]
        { 
            \node (TLS) {\(\T{\LS}\)};      &[7mm]  \node (TTLS) {\(\TT{\LS}\)};        &[7mm]                              &[4mm]                                  &[-4mm] \node (TSSL) {\(\StackTSSL\)};      &                                       &   \node (LSCity) {\(\CityLondon\)};   \\
                                            &                                           &                                   &   \node (TSname) {\(\T{\Sname}\)};    \\
            \node (TLSPlusNLS) {\(\T{S}\)}; &       \node (TTLSPlusNLS) {\(\TT{S}\)};   &   \node (TSno) {\(\T{\Sno}\)};    &                                       &       \node (TSSC) {\(\StackTSSC\)};      &                                       &   \node (TCity) {\(\T{City}\)};   \\
                                            &                                           &                                   &                                       &                                           &   \node (TStatus) {\(\T{\Status}\)};  \\
            \node (TNLS) {\(\T{\NLS}\)};    &       \node (TTNLS) {\(\TT{\NLS}\)};      &                                   &                                       &       \node (TSSNL) {\(\StackTSSNL\)};    &                                       &   \node (NLSCity) {\(\StackTCityMinusLondon\)}; \\
        };
        
        \graph{
            % relation types to tuple types
            { [edges=surtotal]
                (TLS)        -- (TTLS),
                (TLSPlusNLS) -- (TTLSPlusNLS),
                (TNLS)       -- (TTNLS),
            };
            
            % FDs
            (TSno) -- [funcdep left,bend left] (TSSL);
            (TSno) -- [funcdep right] (TSSNL);
            
            % projection edges
            { [edges=projection left]
                (TTLS)  -- (TSno),
                (TTLS)  -- (TSSL),
                (TSSL)  -- { (TStatus), (LSCity) },
                (TTLSPlusNLS) -- (TSno),
            };
            
            { [edges=projection right]
                (TSSC)  -- { (TSname),  (TStatus) },
                (TSSL)  -- (TSname),
                (TSSNL) -- { (TStatus), (NLSCity), (TSname) },
                (TTNLS)  -- (TSno),
                (TTNLS)  -- [arrows={:->}] (TSSNL),
            };
            
            % selection edges
            { [edges=selection left]
                (TSSC)  -- { (TSSL), (TSSNL) },
                (TCity) -- (NLSCity),
            };
            { [edges=selection right]
                (TCity) -- (LSCity),
            };
            
            { [edges=selection right]
                { (TLSPlusNLS), (TTLSPlusNLS) } -- { (TLS), (TTLS) },
            };
            
            { [edges=selection left]
                { (TLSPlusNLS), (TTLSPlusNLS) } -- { (TNLS), (TTNLS) },
            };
            
            % insert "gaps" into crossed edges
            { [edges={white,line width=1.5mm}]
                (TSno) -- (TSSC),
                (TSSC) -- (TCity),
                (TTLSPlusNLS) -- [bend right] (TSSC),
            };
            (TSno) -- [funcdep right] (TSSC);
            (TSSC) -- [projection left] (TCity);
            (TTLSPlusNLS) -- [projection right,arrows={:{crossbar}->},bend right] (TSSC),
        };
        
        \begin{scope}[circle,output]
            \node[draw] at (TLSPlusNLS) [above=10pt] {};
            \node[draw] at (TLSPlusNLS) [below=10pt] {};
            \node[draw] at (TTLSPlusNLS) [above=10pt] {};
            \node[draw] at (TTLSPlusNLS) [below=10pt] {};
        \end{scope}
    \end{tikzpicture}
    \caption{SIG for transformed schema \(S_{3}'\): delete totality annotations on \(\T{S} \protect\TrivialSelection \T{\LS}\), \(\TT{S} \protect\TrivialSelection \TT{\LS}\), \(\T{S} \protect\TrivialSelection \T{\NLS}\), and \(\TT{S} \protect\TrivialSelection \TT{\NLS}\). [\(\preceq\)]}
%     \label{fig-transform-edge-moves}
\end{figure*}

%%%%%%%%%%%%%%%%%%%%

%%%%%%%%%%%%%%%%%%%%

\begin{figure*}
    \centering
    \begin{tikzpicture}
        \matrix[row sep=0.5cm]
        { 
            \node (TLS) {\(\T{\LS}\)};      &[7mm]  \node (TTLS) {\(\TT{\LS}\)};        &[7mm]                              &[4mm]                                  &[-4mm] \node (TSSL) {\(\StackTSSL\)};      &                                       &   \node (LSCity) {\(\CityLondon\)};   \\
                                            &                                           &                                   &   \node (TSname) {\(\T{\Sname}\)};    \\
            \node (TLSPlusNLS) {\(\T{S}\)}; &       \node (TTLSPlusNLS) {\(\TT{S}\)};   &   \node (TSno) {\(\T{\Sno}\)};    &                                       &       \node (TSSC) {\(\StackTSSC\)};      &                                       &   \node (TCity) {\(\T{City}\)};   \\
                                            &                                           &                                   &                                       &                                           &   \node (TStatus) {\(\T{\Status}\)};  \\
            \node (TNLS) {\(\T{\NLS}\)};    &       \node (TTNLS) {\(\TT{\NLS}\)};      &                                   &                                       &       \node (TSSNL) {\(\StackTSSNL\)};    &                                       &   \node (NLSCity) {\(\StackTCityMinusLondon\)}; \\
        };
        
        \graph{
            % relation types to tuple types
            { [edges=surtotal]
                (TLS)        -- (TTLS),
                (TLSPlusNLS) -- (TTLSPlusNLS),
                (TNLS)       -- (TTNLS),
            };
            
            % FDs
            (TSno) -- [funcdep left,bend left] (TSSL);
            (TSno) -- [funcdep right] (TSSNL);
            
            % projection edges
            { [edges=projection left]
                (TTLS)  -- (TSno),
                (TTLS)  -- (TSSL),
                (TSSL)  -- { (TStatus), (LSCity) },
                (TTLSPlusNLS) -- (TSno),
            };
            
            { [edges=projection right]
                (TSSC)  -- { (TSname),  (TStatus) },
                (TSSL)  -- (TSname),
                (TSSNL) -- { (TStatus), (NLSCity), (TSname) },
                (TTNLS)  -- (TSno),
                (TTNLS)  -- [arrows={:->}] (TSSNL),
            };
            
            % selection edges
            { [edges=selection left]
                (TSSC)  -- { (TSSL), (TSSNL) },
                (TCity) -- (NLSCity),
            };
            { [edges=selection right]
                (TCity) -- (LSCity),
            };
            
            { [edges=selection right]
                { (TLSPlusNLS), (TTLSPlusNLS) } -- { (TLS), (TTLS) },
            };
            
            { [edges=selection left]
                { (TLSPlusNLS), (TTLSPlusNLS) } -- { (TNLS), (TTNLS) },
            };
            
            % insert "gaps" into crossed edges
            { [edges={white,line width=1.5mm}]
                (TSno) -- (TSSC),
                (TSSC) -- (TCity),
                (TTLSPlusNLS) -- [bend right] (TSSC),
            };
            (TSno) -- [funcdep right] (TSSC);
            (TSSC) -- [projection left] (TCity);
            (TTLSPlusNLS) -- [projection right,arrows={:{crossbar}->},bend right] (TSSC),
        };
    \end{tikzpicture}
    \caption{Final SIG for transformed schema \(S_{3}'\).}
%     \label{fig-transform-edge-moves}
\end{figure*}

%%%%%%%%%%%%%%%%%%%%

\end{document}
