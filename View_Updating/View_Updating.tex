\documentclass{beamer}


\usepackage[no-math]{fontspec}
\usepackage{mathspec}
\usepackage{xunicode}
\usepackage{xltxtra}
\usepackage{bm}
\usepackage{tikz}
\usepackage{graphicx}
\usepackage{hyperref}


% TikZ setup
\usetikzlibrary{positioning}
\usetikzlibrary{graphs}
\usetikzlibrary{decorations.pathreplacing}
\usetikzlibrary{calc}
\usetikzlibrary{arrows}


% fontspec setup
\defaultfontfeatures{Mapping=tex-text}
\setprimaryfont{TeX Gyre Pagella}
\setsansfont[Scale=MatchUppercase,BoldFont={Gill Sans}]{Gill Sans Light}
\setmonofont[Scale=MatchLowercase]{Letter Gothic 12 Pitch}

% Hack to prevent digits in hyperlinks from being set in the main font instead of the mono font.
% From http://tex.stackexchange.com/questions/99770/problem-with-digits-in-urls-when-using-mathspec-and-hyperref
% Note: doesn't matter if this is executed multiple times.
\makeatletter
    \DeclareMathSymbol{0}{\mathalpha}{\eu@DigitsArabic@symfont}{`0}
    \DeclareMathSymbol{1}{\mathalpha}{\eu@DigitsArabic@symfont}{`1}
    \DeclareMathSymbol{2}{\mathalpha}{\eu@DigitsArabic@symfont}{`2}
    \DeclareMathSymbol{3}{\mathalpha}{\eu@DigitsArabic@symfont}{`3}
    \DeclareMathSymbol{4}{\mathalpha}{\eu@DigitsArabic@symfont}{`4}
    \DeclareMathSymbol{5}{\mathalpha}{\eu@DigitsArabic@symfont}{`5}
    \DeclareMathSymbol{6}{\mathalpha}{\eu@DigitsArabic@symfont}{`6}
    \DeclareMathSymbol{7}{\mathalpha}{\eu@DigitsArabic@symfont}{`7}
    \DeclareMathSymbol{8}{\mathalpha}{\eu@DigitsArabic@symfont}{`8}
    \DeclareMathSymbol{9}{\mathalpha}{\eu@DigitsArabic@symfont}{`9}
\makeatother


% graphicx setup
\graphicspath{{images/}}


% custom macros
% "empty" arrow tip
\pgfarrowsdeclare{:}{:}{}{}
% custom bar arrow tip, offset from end of line (use "empty" tip at line ends if no >)
\tikzset{crossbar/.tip={|[scale width=1.75,sep=0.25em]}}
% various edge styles for TikZ
\tikzset{
    function/.style={arrows={->}},
    injection/.style={arrows={<-}},
    total/.style={arrows={:{crossbar}-}},
    surjection/.style={arrows={-{crossbar}:}},
    bijection/.style={arrows={<{crossbar}-{crossbar}>}},
    projection/.style={arrows={:{crossbar}-{crossbar}>}},
    projection left/.style={arrows={:{crossbar}-{crossbar}>},edge label={\scriptsize\(\RelProject\)}},
    projection right/.style={arrows={:{crossbar}-{crossbar}>},edge label'={\scriptsize\(\RelProject\)}},
    selection left/.style={arrows={<-{crossbar}>},edge label={\scriptsize\(\RelRestrict\)}},
    selection right/.style={arrows={<-{crossbar}>},edge label'={\scriptsize\(\RelRestrict\)}},
    funcdep left/.style={arrows={->},edge node={node[sloped,midway,above] {\scriptsize\emph{key}}}},
    funcdep right/.style={arrows={->},edge node={node[sloped,midway,below] {\scriptsize\emph{key}}}},
    surtotal/.style={arrows={:{crossbar}-{crossbar}:}},
    input keep/.style={blue,thick},
    input delete/.style={blue!40,thick,dashed},
    output/.style={red,thick},
    output temp/.style={red,thick,dashed},
    path 1/.style={green!60!black,thick},
    path 2/.style={orange,thick}
}
        
% projection and selection edge annotations for TikZ
\newcommand{\ProjectionAnnotation}[3][]{%
    \path (#2) to node[above,#1] {\scriptsize\(\RelProject\)} (#3);%
}
\newcommand{\SelectionAnnotation}[3][]{%
    \path (#2) to node[above,#1] {\scriptsize\(\RelRestrict\)} (#3);%
}


\newcounter{constraint}

% misc
\newcommand{\todo}[1]{\textbf{!!TODO!!} {[#1]}}

% commonly used elements
\newcommand{\LS}{\ensuremath{\mathit{LS}}}
\newcommand{\NLS}{\ensuremath{\mathit{NLS}}}
\newcommand{\LSsub}{\ensuremath{\mathit{L}}}
\newcommand{\NLSsub}{\ensuremath{\mathit{N}}}
\newcommand{\Sno}{\ensuremath{\mathit{Sno}}}
\newcommand{\Sname}{\ensuremath{\mathit{Sname}}}
\newcommand{\Status}{\ensuremath{\mathit{Status}}}
\newcommand{\City}{\ensuremath{\mathit{City}}}

\newcommand{\Type}[1]{\ensuremath{T_{#1}}}
\newcommand{\TT}[1]{\ensuremath{T_{\{#1\}}}}

\newcommand{\CityLondon}{\ensuremath{\{\City\colon\allowbreak\mathit{'London'}\}}}
\newcommand{\TCityMinusLondon}{\ensuremath{\Type{\City} \setminus \CityLondon}}
\newcommand{\TSSC}{\ensuremath{\Type{\Sname} \times \Type{\Status} \times \Type{\City}}}
\newcommand{\TSSL}{\ensuremath{\Type{\Sname} \times \Type{\Status} \times \CityLondon}}
\newcommand{\TSSNL}{\ensuremath{\Type{\Sname} \times \Type{\Status} \times (\TCityMinusLondon)}}

\newcommand{\TLSPlusNLS}{\ensuremath{\Type{\LS} + \Type{\NLS}}}
\newcommand{\TTLSPlusNLS}{\ensuremath{\TT{\LS} + \TT{\NLS}}}
\newcommand{\TLSPlusNLSsub}{\ensuremath{\Type{\LSsub} + \Type{\NLSsub}}}
\newcommand{\TTLSPlusNLSsub}{\ensuremath{\TT{\LSsub} + \TT{\NLSsub}}}

\newcommand{\StackTLSPlusNLS}{\ensuremath{\begin{array}{c}\Type{\LS}\,+ \\ \Type{\NLS}\end{array}}}
\newcommand{\StackTTLSPlusNLS}{\ensuremath{\begin{array}{c}\TT{\LS}\,+ \\ \TT{\NLS}\end{array}}}
\newcommand{\StackTLSPlusNLSsub}{\ensuremath{\begin{array}{c}\Type{\LSsub}\,+ \\ \Type{\NLSsub}\end{array}}}
\newcommand{\StackTTLSPlusNLSsub}{\ensuremath{\begin{array}{c}\TT{\LSsub}\,+ \\ \TT{\NLSsub}\end{array}}}
\newcommand{\StackTSSC}{\ensuremath{\begin{array}{c}\Type{\Sname}\,\times \\ \Type{\Status} \times \Type{\City}\end{array}}}
\newcommand{\StackTSSL}{\ensuremath{\begin{array}{c}\Type{\Sname} \times \Type{\Status}\,\times \\ \CityLondon\end{array}}}
\newcommand{\StackTSSNL}{\ensuremath{\begin{array}{c}\Type{\Sname} \times \Type{\Status}\,\times \\ (\TCityMinusLondon)\end{array}}}
\newcommand{\StackTCityMinusLondon}{\ensuremath{\begin{array}{c}\Type{\City}\,\setminus \\ \CityLondon\end{array}}}

\newcommand{\SC}[1]{\ensuremath{\mathcal{S}_{#1}}}

% dominates
\newcommand{\Dominates}[2]{\ensuremath{#2 \preceq #1}}
\newcommand{\Equivalent}[2]{\ensuremath{#1 \equiv #2}}

% SIG notation (in text)
\newcommand{\Sedge}[1]{\ensuremath{\sigma_{\textrm{#1}}}}
\newcommand{\SedgeP}[1]{\ensuremath{\sigma_{\textrm{#1}}^{'}}}

\newcommand{\medmid}{\raise.125ex\hbox{\scalebox{1}[0.75]{$\mid$}}}

% General SIG edges for use in formulas.
% Adapted from: http://tex.stackexchange.com/questions/96330/adding-symbols-at-the-ends-of-a-horizontal-line
\makeatletter
\newlength{\@annotskipleft}
\newlength{\@annotskipright}
% #1 = left edge component
% #2 = right edge component
% #3 = left bar annotation
% #4 = right bar annotation
\newcommand\@sig@edge[4]{%
    \let\@middle\joinrel%
    \ifx#1\relbar%
        \@annotskipleft=.3em%
        % scrunch up the bare line so it's similar length to \long...arrow
        \ifx#2\relbar\def\@middle{\joinrel\joinrel\relbar\joinrel\joinrel}\fi%
    \else% 
        % arrows need a little more clearance
        \@annotskipleft=.4em%
    \fi%
    \ifx#2\relbar\@annotskipright=.3em\else\@annotskipright=.4em\fi%
    \mathrel{\ooalign{$#1\@middle#2$\cr\hskip\@annotskipleft$#3$\hfil$#4$\hskip\@annotskipright\cr}}%
}

% 0 = nothing
% 1 = bar
% 2 = arrowhead
% 3 = both
\def\@sigedge#1#2{%
    \ifcase\numexpr#1*4+#2\relax%
        \@sig@edge{\relbar}{\relbar}{}{}\or                     % 00 = -----
        \@sig@edge{\relbar}{\relbar}{}{\medmid}\or              % 01 = ---+-
        \@sig@edge{\relbar}{\rightarrow}{}{}\or                 % 02 = ---->
        \@sig@edge{\relbar}{\rightarrow}{}{\medmid}\or          % 03 = ---+>
        \@sig@edge{\relbar}{\relbar}{\medmid}{}\or              % 10 = -+---
        \@sig@edge{\relbar}{\relbar}{\medmid}{\medmid}\or       % 11 = -+-+-
        \@sig@edge{\relbar}{\rightarrow}{\medmid}{}\or          % 12 = -+-->
        \@sig@edge{\relbar}{\rightarrow}{\medmid}{\medmid}\or   % 13 = -+-+>
        \@sig@edge{\leftarrow}{\relbar}{}{}\or                  % 20 = <----
        \@sig@edge{\leftarrow}{\relbar}{}{\medmid}\or           % 21 = <--+-
        \@sig@edge{\leftarrow}{\rightarrow}{}{}\or              % 22 = <--->
        \@sig@edge{\leftarrow}{\rightarrow}{}{\medmid}\or       % 23 = <--+>
        \@sig@edge{\leftarrow}{\relbar}{\medmid}{}\or           % 30 = <+---
        \@sig@edge{\leftarrow}{\relbar}{\medmid}{\medmid}\or    % 31 = <+-+-
        \@sig@edge{\leftarrow}{\rightarrow}{\medmid}{}\or       % 32 = <+-->
        \@sig@edge{\leftarrow}{\rightarrow}{\medmid}{\medmid}   % 33 = <+-+>
    \fi%
}

\newcommand{\sigedge}[1]{\ensuremath{\@sigedge#1}}
\makeatother

\newcommand{\Edge}{\sigedge{00}}
\newcommand{\Total}{\sigedge{10}}
\newcommand{\Surjective}{\sigedge{01}}
\newcommand{\SurTotal}{\sigedge{11}}
\newcommand{\Functional}{\sigedge{02}}
\newcommand{\Injective}{\sigedge{20}}
\newcommand{\Bijective}{\sigedge{33}}

% Hack to get the overset symbols to appear at the right height.
% \smash removes the spacing around the operator, hence \mathop.
\newcommand{\LabelledEdge}[2]{\mathop{\overset{\raisebox{0.3em}{\scriptsize\ensuremath{#2}}}{\smash[t]{#1}}}}
\newcommand{\ProjectionEdge}{\LabelledEdge{\sigedge{13}}{\RelProject}}
\newcommand{\SelectionEdge}{\LabelledEdge{\sigedge{23}}{\RelRestrict}}
\newcommand{\TrivialProjection}{\ensuremath{\LabelledEdge{\Bijective}{\RelProject}}}
\newcommand{\TrivialSelection}{\ensuremath{\LabelledEdge{\Bijective}{\RelRestrict}}}
\newcommand{\KeyEdge}{\ensuremath{\LabelledEdge{\Functional}{\mathit{key}}}}

% Constraints.
\newcommand{\Constraint}[2][]{C\ensuremath{_{#2}\ifx&#1&\else^{#1}\fi}}
\newenvironment{ConstraintList}[1][]{%
    \begin{list}{%
        \bfseries%
        \ifx&#1&%
            \Constraint{\ensuremath{\bm{\arabic{constraint}}}}%
        \else%
            \Constraint[\ensuremath{\bm{#1}}]{\ensuremath{\bm{\arabic{constraint}}}}%
        \fi%
    }%
    {\usecounter{constraint}}%
}{\end{list}}


% Draw a grid to aid TikZ picture drawing/debugging.
\newcommand{\DrawGridTikZ}[2]{%
	\begin{scope}[color=lightgray]
		\draw[thin,step=1mm]  (0.0,0.0)   grid (#1,#2);%
		\draw[thick,step=1cm] (-0.1,-0.1) grid (#1+0.1,#2+0.1);%
		\pgftext[top,at={\pgfxy(0.0,-0.2)}]{\tiny 0}%
		\pgftext[right,at={\pgfxy(-0.2,0.0)}]{\tiny 0}%
		\foreach \x in {1,...,#1} {\pgftext[top,at={\pgfxy(\x,-0.2)}]{\tiny\x}}%
		\foreach \y in {1,...,#2} {\pgftext[right,at={\pgfxy(-0.2,\y)}]{\tiny\y}}%
	\end{scope}
}


% Sometimes we want to put a comment in tiny text on the next line, but the default line skip
% will insert too much vertical space. Put a \tinyskip at the end of the line instead.
\def\tinyskip{\\[-0.33\baselineskip]}


% Bold-face text using the structure colour.
\newcommand<>{\structurebf}[1]{\structure#2{\textbf{#1}}}


% preamble
\title{TITLE GOES HERE}
\author{Nigel Stanger \\ \footnotesize Information Science}
\date{August 19, 2016}


\begin{document}


%%%%%%%%%%%%%%%%%%%%%%%%%%%%%%%%%%%%%%%%%%%%%%%%%%%%%%%%%%%%%%%%%%%%%%%%%%%%%%%%


\frame{\titlepage}


%%%%%%%%%%%%%%%%%%%%%%%%%%%%%%%%%%%%%%%%%%%%%%%%%%%%%%%%%%%%%%%%%%%%%%%%%%%%%%%%


\section{The Problem}


%%%%%%%%%%%%%%%%%%%%%%%%%%%%%%%%%%%%%%%%%%%%%%%%%%%%%%%%%%%%%%%%%%%%%%%%%%%%%%%%


\frame{\tableofcontents}


%%%%%%%%%%%%%%%%%%%%%%%%%%%%%%%%%%%%%%%%%%%%%%%%%%%%%%%%%%%%%%%%%%%%%%%%%%%%%%%%


\frame
{
    \centering
    \begin{tikzpicture}
        \node (book) {\includegraphics[height=8cm,keepaspectratio]{Date-ViewUpdatingandRelationalTheory-large.png}};
        \node[anchor=north] (url) at (book.south) {\tiny\href{http://shop.oreilly.com/product/0636920028437.do}{http://shop.oreilly.com/product/0636920028437.do} (2013)};
        \node (date) [below=0.8cm of book.east] {\includegraphics[height=2.2cm,keepaspectratio]{date.png}};
        \only<2->{\fill[fill,yellow,semitransparent] (-0.25,-0.75) rectangle (2.55,-0.4);}
%         \DrawGridTikZ{7.0}{8.0}
    \end{tikzpicture}
}


%%%%%%%%%%%%%%%%%%%%%%%%%%%%%%%%%%%%%%%%%%%%%%%%%%%%%%%%%%%%%%%%%%%%%%%%%%%%%%%%


\frame
{
    \frametitle{Some quick terminology}
    
    \begin{columns}
        \begin{column}{0.33\textwidth}
            \centering
            \includegraphics[width=\columnwidth,keepaspectratio]{Date-SQLandRelationalTheory.png}
            
            \note<1>[item]{This is the second edition; the third edition is now available.}
        \end{column}
        \begin{column}{0.67\textwidth}
            \structurebf{Value vs.\ variable}
            \begin{itemize}
                \item a value is immutable (“2”)
                \item a variable contains a value
                \item[\(\Rightarrow\)] \emph{relation} value, variable (\emph{relvar})
            \end{itemize}
            \medskip
            
            \structurebf{Relation \alert<2>{heading} \& \alert<3>{body}}
            \medskip
            
            {\scriptsize
            \begin{tabular}{c|r|l|l|l|}
                \cline{2-5}
                \textbf{S}  &   \textbf{\alert<2>{Sno}} &   \textbf{\alert<2>{Sname}}   &   \textbf{\alert<2>{Status}}  &   \textbf{\alert<2>{City}}    \\
                \cline{2-5}
                            &   \alert<3>{1}            &   \alert<3>{Smith}            &   \alert<3>{20}               &   \alert<3>{London}   \\
                            &   \alert<3>{2}            &   \alert<3>{Jones}            &   \alert<3>{40}               &   \alert<3>{Paris}    \\
                \cline{2-5}
            \end{tabular}}
            \bigskip
            
            \structurebf{Type (\(\bm{\Type{x}}\))}
            \begin{itemize}
                \item (finite) set of values that something can take
            \end{itemize}
            \medskip
            
            \structurebf{Date doesn’t believe in nulls}\tinyskip
            {\tiny (but it doesn’t really matter here)}
        \end{column}
    \end{columns}
}


%%%%%%%%%%%%%%%%%%%%%%%%%%%%%%%%%%%%%%%%%%%%%%%%%%%%%%%%%%%%%%%%%%%%%%%%%%%%%%%%


\frame
{
    \frametitle{Some quick terminology}
    
    \structurebf{Attribute type (\(\bm{\Type{A}}\))}
    \begin{itemize}
        \item the set of all possible values that \(A\) can take
    \end{itemize}
    \medskip
    
    \structurebf{Relation type (\(\bm{\Type{R}}\))}
    \begin{itemize}
        \item the set of all possible relation values with \(R\)’s heading
    \end{itemize}
    \medskip
    
    \structurebf{Tuple type (\(\bm{\TT{R}}\))}
    \begin{itemize}
        \item the set of all possible tuple values with \(R\)’s heading
        \item[=] Cartesian product of \(R\)’s attribute types
        \item[=] \(\Type{A_{1}} \times \Type{A_{2}} \times \dotsc \times \Type{A_{n}}\)
    \end{itemize}
}


%%%%%%%%%%%%%%%%%%%%%%%%%%%%%%%%%%%%%%%%%%%%%%%%%%%%%%%%%%%%%%%%%%%%%%%%%%%%%%%%


\frame
{
    \frametitle{Date’s (hoary and clichéd) suppliers-and-parts schema}
    
    \(S\{\Sno, \Sname, \Status, \City\}\)
    
    \quad “Supplier \(\Sno\) has name \(\Sname\), status \(\Status\) and city \(\City\)”
    \bigskip
    
    \(P\{\mathit{Pno}, \mathit{Pname}, \mathit{Color}, \City\}\)
    
    \quad “Part …”
    \bigskip
    
    \(SP\{\Sno, \mathit{Pno}, \mathit{Qty}\}\)
    
    \quad “Shipment …”
}


%%%%%%%%%%%%%%%%%%%%%%%%%%%%%%%%%%%%%%%%%%%%%%%%%%%%%%%%%%%%%%%%%%%%%%%%%%%%%%%%


%%%%%%%%%%%%%%%%%%%%%%%%%%%%%%%%%%%%%%%%%%%%%%%%%%%%%%%%%%%%%%%%%%%%%%%%%%%%%%%%


\end{document}

% outline:
%√ Date’s book
%√ Terminology and conventions
%√    values vs. variables
%√    relational values vs. relvar
%√    relation heading and body
%√    types (domains)
% The view update problem
%    example of ambiguity (join views)
%    previous approaches
%    Date’s approach
%        his pragmatic solution to join view ambiguity
% Date’s informal operational definition of information equivalence
%    can it be characterised more formally?
%    could this reveal missing elements in his analysis?
% Relative information capacity
%    loose definition
% Schema intension graphs
%    basic elements
%    constructing a simple SIG
%√        the hoary suppliers-parts schema
% Capturing more detailed constraint information (what’s missing in SIGs currently?)
%    types
%    (primary) keys
%    foreign keys? (inclusion dependencies)
% Schemas, sub-schemas, and constraint propagation
%    enumeration of all constraints is important
%    rewriting constraints in terms of the sub-schema
% The restriction view example
