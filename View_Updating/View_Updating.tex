\documentclass{beamer}


\usepackage[no-math]{fontspec}
\usepackage{xunicode}
\usepackage{xltxtra}
\usepackage{bm}
\usepackage{tikz}
\usepackage{graphicx}
\usepackage{relalg}
\usepackage{booktabs}
\usepackage{hyperref}


% Beamer setup
\usefonttheme{structurebold}
\useinnertheme{rectangles}
\usenavigationsymbolstemplate{}
\setbeamertemplate{footline}[frame number]


% TikZ setup
\usetikzlibrary{positioning}
\usetikzlibrary{graphs}
\usetikzlibrary{decorations.pathreplacing}
\usetikzlibrary{calc}
\usetikzlibrary{arrows}


% fontspec setup
\defaultfontfeatures{Mapping=tex-text}
\setsansfont[Scale=MatchUppercase,BoldFont={Gill Sans}]{Gill Sans Light}
\setmonofont[Scale=MatchLowercase]{Letter Gothic 12 Pitch}


% graphicx setup
\graphicspath{{images/}}


% custom macros
% "empty" arrow tip
\pgfarrowsdeclare{:}{:}{}{}
% custom bar arrow tip, offset from end of line (use "empty" tip at line ends if no >)
\tikzset{crossbar/.tip={|[scale width=1.75,sep=0.25em]}}
% various edge styles for TikZ
\tikzset{
    function/.style={arrows={->}},
    injection/.style={arrows={<-}},
    total/.style={arrows={:{crossbar}-}},
    surjection/.style={arrows={-{crossbar}:}},
    bijection/.style={arrows={<{crossbar}-{crossbar}>}},
    projection/.style={arrows={:{crossbar}-{crossbar}>}},
    projection left/.style={arrows={:{crossbar}-{crossbar}>},edge label={\scriptsize\(\RelProject\)}},
    projection right/.style={arrows={:{crossbar}-{crossbar}>},edge label'={\scriptsize\(\RelProject\)}},
    selection left/.style={arrows={<-{crossbar}>},edge label={\scriptsize\(\RelRestrict\)}},
    selection right/.style={arrows={<-{crossbar}>},edge label'={\scriptsize\(\RelRestrict\)}},
    funcdep left/.style={arrows={:{crossbar}-{crossbar}>},edge node={node[sloped,midway,above] {\scriptsize\emph{key}}}},
    funcdep right/.style={arrows={:{crossbar}-{crossbar}>},edge node={node[sloped,midway,below] {\scriptsize\emph{key}}}},
    surtotal/.style={arrows={:{crossbar}-{crossbar}:}},
    input keep/.style={structure,thick},
    input delete/.style={structure!40,thick,dashed},
    output/.style={alert,thick},
    output temp/.style={alert,thick,dashed},
    path 1/.style={green!60!black,thick},
    path 2/.style={orange,thick},
    invisible/.style={opacity=0},
    visible on/.style={alt=#1{}{invisible}},
    input on/.style={alt=#1{input keep}{}},
    output on/.style={alt=#1{output}{}},
    alt/.code args={<#1>#2#3}{\alt<#1>{\pgfkeysalso{#2}}{\pgfkeysalso{#3}}}
}
        
% projection and selection edge annotations for TikZ
\newcommand{\ProjectionAnnotation}[3][]{%
    \path (#2) to node[above,#1] {\scriptsize\(\RelProject\)} (#3);%
}
\newcommand{\SelectionAnnotation}[3][]{%
    \path (#2) to node[above,#1] {\scriptsize\(\RelRestrict\)} (#3);%
}


\newcounter{constraint}

% misc
\newcommand{\todo}[1]{\textbf{!!TODO!!} {[#1]}}

% commonly used elements
\newcommand{\LS}{\ensuremath{\mathit{LS}}}
\newcommand{\NLS}{\ensuremath{\mathit{NLS}}}
\newcommand{\LSsub}{\ensuremath{\mathit{L}}}
\newcommand{\NLSsub}{\ensuremath{\mathit{N}}}
\newcommand{\Sno}{\ensuremath{\mathit{Sno}}}
\newcommand{\Sname}{\ensuremath{\mathit{Sname}}}
\newcommand{\Status}{\ensuremath{\mathit{Status}}}
\newcommand{\City}{\ensuremath{\mathit{City}}}

\newcommand{\Type}[1]{\ensuremath{T_{#1}}}
\newcommand{\TT}[1]{\ensuremath{T_{\{#1\}}}}

\newcommand{\CityLondon}{\ensuremath{\{\City\colon\allowbreak\text{`\emph{London}'}\}}}
\newcommand{\TCityMinusLondon}{\ensuremath{\Type{\City} \setminus \CityLondon}}
\newcommand{\TSSC}{\ensuremath{\Type{\Sname} \times \Type{\Status} \times \Type{\City}}}
\newcommand{\TSSL}{\ensuremath{\Type{\Sname} \times \Type{\Status} \times \CityLondon}}
\newcommand{\TSSNL}{\ensuremath{\Type{\Sname} \times \Type{\Status} \times (\TCityMinusLondon)}}

\newcommand{\TLSPlusNLS}{\ensuremath{\Type{\LS} + \Type{\NLS}}}
\newcommand{\TTLSPlusNLS}{\ensuremath{\TT{\LS} + \TT{\NLS}}}
\newcommand{\TLSPlusNLSsub}{\ensuremath{\Type{\LSsub} + \Type{\NLSsub}}}
\newcommand{\TTLSPlusNLSsub}{\ensuremath{\TT{\LSsub} + \TT{\NLSsub}}}

\newcommand{\StackTLSPlusNLS}{\ensuremath{\begin{array}{c}\Type{\LS}\,+ \\ \Type{\NLS}\end{array}}}
\newcommand{\StackTTLSPlusNLS}{\ensuremath{\begin{array}{c}\TT{\LS}\,+ \\ \TT{\NLS}\end{array}}}
\newcommand{\StackTLSPlusNLSsub}{\ensuremath{\begin{array}{c}\Type{\LSsub}\,+ \\ \Type{\NLSsub}\end{array}}}
\newcommand{\StackTTLSPlusNLSsub}{\ensuremath{\begin{array}{c}\TT{\LSsub}\,+ \\ \TT{\NLSsub}\end{array}}}
\newcommand{\StackTSSC}{\ensuremath{\begin{array}{c}\Type{\Sname}\,\times \\ \Type{\Status} \times \Type{\City}\end{array}}}
\newcommand{\StackTSSL}{\ensuremath{\begin{array}{c}\Type{\Sname} \times \Type{\Status}\,\times \\ \CityLondon\end{array}}}
\newcommand{\StackTSSNL}{\ensuremath{\begin{array}{c}\Type{\Sname} \times \Type{\Status}\,\times \\ (\TCityMinusLondon)\end{array}}}
\newcommand{\StackTCityMinusLondon}{\ensuremath{\begin{array}{c}\Type{\City}\,\setminus \\ \CityLondon\end{array}}}

\newcommand{\SC}[1]{\ensuremath{\mathcal{S}_{#1}}}

% dominates
\newcommand{\Dominates}[2]{\ensuremath{#2 \preceq #1}}
\newcommand{\Equivalent}[2]{\ensuremath{#1 \equiv #2}}

% SIG notation (in text)
\newcommand{\Sedge}[1]{\ensuremath{\sigma_{\textrm{#1}}}}
\newcommand{\SedgeP}[1]{\ensuremath{\sigma_{\textrm{#1}}^{'}}}

\newcommand{\medmid}{\raise.125ex\hbox{\scalebox{1}[0.75]{$\mid$}}}

% General SIG edges for use in formulas.
% Adapted from: http://tex.stackexchange.com/questions/96330/adding-symbols-at-the-ends-of-a-horizontal-line
\makeatletter
\newlength{\@annotskipleft}
\newlength{\@annotskipright}
% #1 = left edge component
% #2 = right edge component
% #3 = left bar annotation
% #4 = right bar annotation
\newcommand\@sig@edge[4]{%
    \let\@middle\joinrel%
    \ifx#1\relbar%
        \@annotskipleft=.3em%
        % scrunch up the bare line so it's similar length to \long...arrow
        \ifx#2\relbar\def\@middle{\joinrel\joinrel\relbar\joinrel\joinrel}\fi%
    \else% 
        % arrows need a little more clearance
        \@annotskipleft=.4em%
    \fi%
    \ifx#2\relbar\@annotskipright=.3em\else\@annotskipright=.4em\fi%
    \mathrel{\ooalign{$#1\@middle#2$\cr\hskip\@annotskipleft$#3$\hfil$#4$\hskip\@annotskipright\cr}}%
}

% 0 = nothing
% 1 = bar
% 2 = arrowhead
% 3 = both
\def\@sigedge#1#2{%
    \ifcase\numexpr#1*4+#2\relax%
        \@sig@edge{\relbar}{\relbar}{}{}\or                     % 00 = -----
        \@sig@edge{\relbar}{\relbar}{}{\medmid}\or              % 01 = ---+-
        \@sig@edge{\relbar}{\rightarrow}{}{}\or                 % 02 = ---->
        \@sig@edge{\relbar}{\rightarrow}{}{\medmid}\or          % 03 = ---+>
        \@sig@edge{\relbar}{\relbar}{\medmid}{}\or              % 10 = -+---
        \@sig@edge{\relbar}{\relbar}{\medmid}{\medmid}\or       % 11 = -+-+-
        \@sig@edge{\relbar}{\rightarrow}{\medmid}{}\or          % 12 = -+-->
        \@sig@edge{\relbar}{\rightarrow}{\medmid}{\medmid}\or   % 13 = -+-+>
        \@sig@edge{\leftarrow}{\relbar}{}{}\or                  % 20 = <----
        \@sig@edge{\leftarrow}{\relbar}{}{\medmid}\or           % 21 = <--+-
        \@sig@edge{\leftarrow}{\rightarrow}{}{}\or              % 22 = <--->
        \@sig@edge{\leftarrow}{\rightarrow}{}{\medmid}\or       % 23 = <--+>
        \@sig@edge{\leftarrow}{\relbar}{\medmid}{}\or           % 30 = <+---
        \@sig@edge{\leftarrow}{\relbar}{\medmid}{\medmid}\or    % 31 = <+-+-
        \@sig@edge{\leftarrow}{\rightarrow}{\medmid}{}\or       % 32 = <+-->
        \@sig@edge{\leftarrow}{\rightarrow}{\medmid}{\medmid}   % 33 = <+-+>
    \fi%
}

\newcommand{\sigedge}[1]{\ensuremath{\@sigedge#1}}
\makeatother

\newcommand{\Edge}{\sigedge{00}}
\newcommand{\Total}{\sigedge{10}}
\newcommand{\Surjective}{\sigedge{01}}
\newcommand{\SurTotal}{\sigedge{11}}
\newcommand{\Functional}{\sigedge{02}}
\newcommand{\Injective}{\sigedge{20}}
\newcommand{\Bijective}{\sigedge{33}}

% Hack to get the overset symbols to appear at the right height.
% \smash removes the spacing around the operator, hence \mathop.
\newcommand{\LabelledEdge}[2]{\mathop{\overset{\raisebox{0.3em}{\scriptsize\ensuremath{#2}}}{\smash[t]{#1}}}}
\newcommand{\ProjectionEdge}{\LabelledEdge{\sigedge{13}}{\RelProject}}
\newcommand{\SelectionEdge}{\LabelledEdge{\sigedge{23}}{\RelRestrict}}
\newcommand{\TrivialProjection}{\ensuremath{\LabelledEdge{\Bijective}{\RelProject}}}
\newcommand{\TrivialSelection}{\ensuremath{\LabelledEdge{\Bijective}{\RelRestrict}}}
\newcommand{\KeyEdge}{\ensuremath{\LabelledEdge{\sigedge{13}}{\mathit{key}}}}

% Constraints.
\newcommand{\Constraint}[2][]{C\ensuremath{_{#2}\ifx&#1&\else^{#1}\fi}}
\newenvironment{ConstraintList}[1][]{%
    \begin{list}{%
        \bfseries%
        \ifx&#1&%
            \Constraint{\ensuremath{\bm{\arabic{constraint}}}}%
        \else%
            \Constraint[\ensuremath{\bm{#1}}]{\ensuremath{\bm{\arabic{constraint}}}}%
        \fi%
    }%
    {\usecounter{constraint}}%
}{\end{list}}


% Draw a grid to aid TikZ picture drawing/debugging.
\newcommand{\DrawGridTikZ}[2]{%
	\begin{scope}[color=lightgray]
		\draw[thin,step=1mm]  (0.0,0.0)   grid (#1,#2);%
		\draw[thick,step=1cm] (-0.1,-0.1) grid (#1+0.1,#2+0.1);%
		\pgftext[top,at={\pgfxy(0.0,-0.2)}]{\tiny 0}%
		\pgftext[right,at={\pgfxy(-0.2,0.0)}]{\tiny 0}%
		\foreach \x in {1,...,#1} {\pgftext[top,at={\pgfxy(\x,-0.2)}]{\tiny\x}}%
		\foreach \y in {1,...,#2} {\pgftext[right,at={\pgfxy(-0.2,\y)}]{\tiny\y}}%
	\end{scope}
}


% Sometimes we want to put a comment in tiny text on the next line, but the default line skip
% will insert too much vertical space. Put a \tinyskip at the end of the line instead.
\def\tinyskip{\\[-0.33\baselineskip]}


% Bold-face text using the structure colour.
\newcommand<>{\structurebf}[1]{\structure#2{\textbf{#1}}}


% preamble
\title{Characterising relational view updates \\ using relative information capacity}
\author{Nigel Stanger \\ \footnotesize Information Science}
\date{August 19, 2016}


\begin{document}


%%%%%%%%%%%%%%%%%%%%%%%%%%%%%%%%%%%%%%%%%%%%%%%%%%%%%%%%%%%%%%%%%%%%%%%%%%%%%%%%


\frame
{
    \thispagestyle{empty}
    \titlepage
}


%%%%%%%%%%%%%%%%%%%%%%%%%%%%%%%%%%%%%%%%%%%%%%%%%%%%%%%%%%%%%%%%%%%%%%%%%%%%%%%%


\frame
{
    \centering
    \begin{tikzpicture}
        \node (book) {\includegraphics[height=8cm,keepaspectratio]{Date-ViewUpdatingandRelationalTheory-large.png}};
        \node[anchor=north] (url) at (book.south) {\tiny\href{http://shop.oreilly.com/product/0636920028437.do}{http://shop.oreilly.com/product/0636920028437.do} (2013)};
        \node (date) [below=0.8cm of book.east] {\includegraphics[height=2.2cm,keepaspectratio]{date.png}};
        \only<2->{\fill[fill,yellow,semitransparent] (-0.25,-0.75) rectangle (2.55,-0.4);}
%         \DrawGridTikZ{7.0}{8.0}
    \end{tikzpicture}
}


%%%%%%%%%%%%%%%%%%%%%%%%%%%%%%%%%%%%%%%%%%%%%%%%%%%%%%%%%%%%%%%%%%%%%%%%%%%%%%%%


\frame
{
    \frametitle{Some quick terminology}
    
    \begin{columns}
        \begin{column}{0.4\textwidth}
            \centering
            \includegraphics[width=0.95\columnwidth,keepaspectratio]{Date-SQLandRelationalTheory.png}
            
            \tiny\href{http://shop.oreilly.com/product/0636920022879.do}{http://shop.oreilly.com/product/0636920022879.do} \\
            (second edition, 2011)
            
            \note<1>[item]{This is the second edition; the third edition is now available.}
        \end{column}
        \begin{column}{0.6\textwidth}
            \structurebf{Value vs.\ variable}
            \begin{itemize}
                \item a value is immutable (“2”)
                \item a variable contains a value
                \item[\(\Rightarrow\)] \emph{relation} value, variable (\emph{relvar})
            \end{itemize}
            \medskip
            
            \structurebf{Relation \alert<2>{heading} \& \alert<3>{body}}
            \medskip
            
            {\scriptsize
            \begin{tabular}{c|l|l|r|l|}
                \cline{2-5}
                \textbf{S}  &   \textbf{\alert<2>{Sno}} &   \textbf{\alert<2>{Sname}}   &   \textbf{\alert<2>{Status}}  &   \textbf{\alert<2>{City}}    \\
                \cline{2-5}
                            &   \alert<3>{S1}           &   \alert<3>{Smith}            &   \alert<3>{20}               &   \alert<3>{London}   \\
                            &   \alert<3>{S2}           &   \alert<3>{Jones}            &   \alert<3>{30}               &   \alert<3>{Paris}    \\
                \cline{2-5}
            \end{tabular}}
            \bigskip
            
            \structurebf{Type (\(\bm{\Type{x}}\))}
            \begin{itemize}
                \item (finite) set of values that something (\(x\)) can take
            \end{itemize}
            \medskip
            
            \structurebf{Date doesn’t believe in nulls}\tinyskip
            {\tiny (but it doesn’t really matter here)}
        \end{column}
    \end{columns}
}


%%%%%%%%%%%%%%%%%%%%%%%%%%%%%%%%%%%%%%%%%%%%%%%%%%%%%%%%%%%%%%%%%%%%%%%%%%%%%%%%


\frame
{
    \frametitle{More on types}
    
    \structurebf{Attribute type (\(\bm{\Type{A}}\))}
    \begin{itemize}
        \item set of all possible values that \(A\) can take
        \item[=] \(\{v_{1}\text{,}\, v_{2}\text{,}\, \dotsc\text{,}\, v_{m}\}\) {\tiny (for finite \(m\))}
    \end{itemize}
    \medskip
    
    \structurebf{Tuple type (\(\bm{\TT{R}}\)) of relvar \(\bm{R}\)}
    \begin{itemize}
        \item set of all possible tuple values with \(R\)’s heading
        \item[=] Cartesian product of \(R\)’s attribute types
        \item[=] \(\Type{A_{1}} \times \Type{A_{2}} \times \dotsb \times \Type{A_{n}}\)
    \end{itemize}
    \medskip
    
    \structurebf{Relation type (\(\bm{\Type{R}}\)) of relvar \(\bm{R}\)}
    \begin{itemize}
        \item set of all possible relation values with \(R\)’s heading
        \item[=] every possible combination of elements of \(\TT{R}\)
    \end{itemize}
}


%%%%%%%%%%%%%%%%%%%%%%%%%%%%%%%%%%%%%%%%%%%%%%%%%%%%%%%%%%%%%%%%%%%%%%%%%%%%%%%%


\frame
{
    \frametitle{Date’s (hoary and clichéd) suppliers-and-parts schema}
    
    \structurebf{\(\bm{S\{\Sno\text{,}\, \Sname\text{,}\, \Status\text{,}\, \City\}}\)}
    
    \begin{quote}
        \upshape
        Supplier \(\Sno\) is under contract, is named \(\Sname\), has status \(\Status\), and is located in city \(\City\)
    \end{quote}
    \bigskip
    
    \structurebf{\(\bm{P\{\mathit{Pno}\text{,}\, \mathit{Pname}\text{,}\, \mathit{Colour}\text{,}\, \mathit{Weight}\text{,}\, \City\}}\)}
    
    \begin{quote}
        \upshape
        Part \(\mathit{Pno}\) is used in the enterprise, is named \(\mathit{Pname}\), has colour \(\mathit{Colour}\), and weight \(\mathit{Weight}\)
    \end{quote}
    \bigskip
    
    \structurebf{\(\bm{SP\{\Sno\text{,}\, \mathit{Pno}\text{,}\, \mathit{Qty}\}}\)}
    
    \begin{quote}
        \upshape
        Supplier \(\Sno\) supplies (ships) part \(\mathit{Pno}\) in quantity \(\mathit{Qty}\)
    \end{quote}
}


%%%%%%%%%%%%%%%%%%%%%%%%%%%%%%%%%%%%%%%%%%%%%%%%%%%%%%%%%%%%%%%%%%%%%%%%%%%%%%%%


\frame
{
    \frametitle{The view update problem}

    \centering
    \Large
    \begin{tikzpicture}[every edge/.style={draw,semithick}]
        \node (vdb)                         {\(V(\mathit{DB})\)};
        \node (db)   [below=18mm of vdb]    {\(\mathit{DB}\vphantom{(')}\)};

        \node (vdbp) [right=4cm of vdb]     {\(V(\mathit{DB}')\)};
        \node (uvdb) [left=-1mm of vdbp]    {\(U(V(\mathit{DB})) \stackrel{?}{=}\)};

        \node (dbp)  at (db -| vdbp)        {\(\mathit{DB}\smash[t]{'}\vphantom{()}\)};
        \node (tudb) [left=-1mm of dbp]     {\(T(U)(\mathit{DB}) =\)};

        \graph{
            { (db), (dbp) } -> [edge label={\(V\)},swap] { (vdb), (vdbp) };

            (vdb) -> [edge node={node (u)  [above=-0.6mm] {\(U\)}}]    (uvdb);
            (db)  -> [edge node={node (tu) [above=-0.6mm] {\(T(U)\)}}] (tudb);

            (u)   -> [double equal sign distance=2pt,-implies,edge label={\(T\)}] (tu);
        };
    \end{tikzpicture}
    \\
    \tiny Keller (1985) \emph{Algorithms for translating view updates to database updates for views involving selections, projections, and joins}
    
    \note<1>{Keller: “The user specifies update \(U\) against the view of the database, \(V(\mathit{DB})\). The view update translator \(T\) supplies the database update \(T(U)\), which results in \(DB'\) when applied to the database. The new view state is \(V(\mathit{DB}')\). This translation has no side effects in the view if \(V(\mathit{DB}') = U(V(\mathit{DB}))\), that is, if the view has changed precisely in accordance with the user’s request.”}
    
    \bigskip
    \normalsize
    
    \begin{uncoverenv}<2->
        \structurebf{Problem:} there are often multiple update translations \(T(U)\) that produce the effect of \(U\) but have quite different effects on \(\mathit{DB}\).
        \bigskip
    
        How do we choose?
    \end{uncoverenv}
}


%%%%%%%%%%%%%%%%%%%%%%%%%%%%%%%%%%%%%%%%%%%%%%%%%%%%%%%%%%%%%%%%%%%%%%%%%%%%%%%%


\frame
{
    \frametitle{The view update problem}
    
    \begin{columns}
        \begin{column}[T]{0.45\textwidth}
            \footnotesize\color{gray!70}
            \begin{tabular}{|l|l|r|l|}
                \multicolumn{4}{l}{\textbf{S}}  \\
                \hline
                \textbf{Sno}        &   \textbf{Sname}      &   \textbf{Status}     &   \textbf{City}   \\
                \hline
                S1                  & Smith                 & 20                    & London \\
                S2                  & Jones                 & 30                    & Paris \\
                S3                  & Blake                 & 30                    & Paris \\
                \only<1-2,4->{S4}  & \only<1-2,4->{Clark} & \only<1-2,4->{20}    & \only<1-2,4,6>{London}\only<5>{\alert{???}}\only<7->{\alert{Paris}} \\
                S5                  & Adams                 & 30                    & Athens \\
                \hline
            \end{tabular}
        \end{column}
        \begin{column}[T]{0.45\textwidth}
            \footnotesize
            \begin{tabular}{|l|l|r|l|}
                \multicolumn{4}{l}{\(\bm{\text{\textbf{LS}} = \RelRestrict_{\text{\textbf{City}} = \text{\textbf{`London'}}} \,\text{\textbf{S}}}\)}  \\
                \hline
                \textbf{Sno}    &   \textbf{Sname}  &   \textbf{Status} &   \textbf{City}   \\
                \hline
                S1              & Smith             & 20                & London \\
                \only<1-2,4,6>{S4} & \only<1-2,4,6>{Clark}& \only<1-2,4,6>{20}   & \only<1-2,4,6>{London} \\
                \hline
            \end{tabular}
        \end{column}
    \end{columns}
    \bigskip
    
    \structurebf{Delete S4 from LS?}
    
    \begin{enumerate}
        \item<2-> Delete S4 from S.
        \item<4-> Change S4’s city to anything other than London.
    \end{enumerate}
    \bigskip
    
    \begin{uncoverenv}<6->
        \structurebf{What if we instead \emph{change} S4’s city to Paris in LS?}
        \begin{itemize}
            \item<7->[\(\Rightarrow\)] changing S4’s city to Paris in S
            \item<7-> side effect: some changes behave like deletes!
        \end{itemize}
    \end{uncoverenv}
}


%%%%%%%%%%%%%%%%%%%%%%%%%%%%%%%%%%%%%%%%%%%%%%%%%%%%%%%%%%%%%%%%%%%%%%%%%%%%%%%%


\frame
{
    \frametitle{The view update problem}
    
    \tiny
    \begin{columns}
        \begin{column}[T]{0.35\textwidth}
            \color{gray!70}
            \begin{tabular}{|l|l|r|l|}
                \multicolumn{4}{l}{\textbf{S}}  \\
                \hline
                \textbf{Sno}        &   \textbf{Sname}      &   \textbf{Status}     &   \textbf{City}   \\
                \hline
                \only<1-6,8-9>{S1}\only<7>{\alert{S6}} & \only<1-9>{Smith} & \only<1-9>{20} & \only<1-9>{London} \\
                S2                  & Jones                 & 30                    & Paris \\
                S3                  & Blake                 & 30                    & Paris \\
                S4                  & Clark                 & 20                    & London \\
                S5                  & Adams                 & 30                    & Athens \\
                \hline
            \end{tabular}
            \bigskip
            
            \begin{tabular}{|l|l|r|}
                \multicolumn{3}{l}{\textbf{SP}}  \\
                \hline
                \textbf{Sno}        &   \textbf{Pno}      &   \textbf{Qty}   \\
                \hline
                \only<1-2,4,6,8-9>{S1}\only<5>{\alert{S3}}\only<7>{\alert{S6}} & \only<1-2,4-9>{P1} & \only<1-2,4-9>{300} \\
                S1 & P2 & 200 \\
                S1 & P3 & 400 \\
                S1 & P4 & 200 \\
                S1 & P5 & 100 \\
                S1 & P6 & 100 \\
                S2 & P1 & 300 \\
                S2 & P2 & 400 \\
                S3 & P2 & 200 \\
                S4 & P2 & 300 \\
                S4 & P4 & 300 \\
                S4 & P5 & 400 \\
                \hline
            \end{tabular}
            \bigskip\medskip
            
            \color{black}
            \uncover<8>{\tiny\alert{\href{https://www.youtube.com/watch?v=hou0lU8WMgo}{\beamergotobutton{technically correct}}}}
        \end{column}
        \begin{column}[T]{0.55\textwidth}
            \begin{tabular}{|l|l|r|l|l|r|}
                \multicolumn{4}{l}{\(\bm{\mathsf{SSP} = \mathsf{S} \RelNaturalJoin \mathsf{SP}}\)}  \\
                \hline
                \textbf{Sno}        &   \textbf{Sname}      &   \textbf{Status}     &   \textbf{City}   &   \textbf{Pno}    &   \textbf{Qty}   \\
                \hline
                \only<1-2,4,6,8-9>{S1}\only<5>{\alert{S3}}\only<7>{\alert{S6}} & \only<1-2,4,6-9>{Smith}\only<5>{\alert{Blake}} & \only<1-2,4,6-9>{20}\only<5>{\alert{30}} & \only<1-2,4,6-9>{London}\only<5>{\alert{Paris}} & \only<1-2,4-9>{P1} & \only<1-2,4-9>{300} \\
                \only<1-6,8-9>{S1}\only<7>{\alert{S6}} & \only<1-9>{Smith} & \only<1-9>{20} & \only<1-9>{London} & \only<1-9>{P2} & \only<1-9>{200} \\
                \only<1-6,8-9>{S1}\only<7>{\alert{S6}} & \only<1-9>{Smith} & \only<1-9>{20} & \only<1-9>{London} & \only<1-9>{P3} & \only<1-9>{400} \\
                \only<1-6,8-9>{S1}\only<7>{\alert{S6}} & \only<1-9>{Smith} & \only<1-9>{20} & \only<1-9>{London} & \only<1-9>{P4} & \only<1-9>{200} \\
                \only<1-6,8-9>{S1}\only<7>{\alert{S6}} & \only<1-9>{Smith} & \only<1-9>{20} & \only<1-9>{London} & \only<1-9>{P5} & \only<1-9>{100} \\
                \only<1-6,8-9>{S1}\only<7>{\alert{S6}} & \only<1-9>{Smith} & \only<1-9>{20} & \only<1-9>{London} & \only<1-9>{P6} & \only<1-9>{100} \\
                S2 & Jones & 30 & Paris & P1 & 300 \\
                S2 & Jones & 30 & Paris & P2 & 400 \\
                S3 & Blake & 30 & Paris & P2 & 200 \\
                S4 & Clark & 20 & London & P4 & 300 \\
                S4 & Clark & 20 & London & P5 & 400 \\
                S4 & Clark & 20 & London & P2 & 200 \\
                \hline
            \end{tabular}
            \bigskip
            
            \scriptsize
            \structurebf{Delete first row of SSP?}
            
            \begin{enumerate}
                \item<2-> Delete \(\text{SP}\{\text{S1}\text{,}\, \text{P1}\text{,}\, \text{300}\}\).
                \item<4-|alert@8> Change supplier number in \(\text{SP}\{\text{S1}\text{,}\, \text{P1}\text{,}\, \text{300}\}\).
                \item<6-|alert@8> Change S1’s supplier number in S.
                \item<9-> Delete S1 from S. {\tiny (nuclear option)}
            \end{enumerate}
            \smallskip
            
            \uncover<11>{\structurebf{Changes are even more problematic}}
        \end{column}
    \end{columns}
}


%%%%%%%%%%%%%%%%%%%%%%%%%%%%%%%%%%%%%%%%%%%%%%%%%%%%%%%%%%%%%%%%%%%%%%%%%%%%%%%%


\frame
{
    \frametitle{Date's solution}
    
    \begin{itemize}
    
        \item Explicitly specify \emph{all} constraints in advance.
        
        \item Automatically generate \emph{compensatory rules} from constraints.\tinyskip
        {\tiny (kind of like declarative triggers)}
        
        \item No problem if original and view schemas are \emph{information equivalent}.\tinyskip
        {\tiny (no hidden information)}
        \begin{itemize}
        
            \item[] ``…if and only if the constraints that apply to [schema \(\SC{1}\)] and [schema \(\SC{2}\)] are such that every proposition that can be represented by [\(\SC{1}\)] can be represented by [\(\SC{2}\)] and vice versa''\\
            \hfill{\tiny Date (2013) \emph{View Updating \& Relational Theory}}
        
        \end{itemize}
        \medskip
        
        \item[BUT:] Still potential for ambiguity when views hide information.\tinyskip
        {\tiny(i.e., no longer information equivalent)}
        \medskip
                
        \item Date’s approach involves “\emph{…a clear element of pragma…}”.\tinyskip
        {\tiny (essentially he argues for “common sense” ambiguity resolution)}
    
    \end{itemize}
}


%%%%%%%%%%%%%%%%%%%%%%%%%%%%%%%%%%%%%%%%%%%%%%%%%%%%%%%%%%%%%%%%%%%%%%%%%%%%%%%%


\frame
{
    \frametitle{Relative information capacity}
    \framesubtitle{}
    
    \begin{itemize}
    
        \item “Information content” of a schema. {\tiny (including constraints)}
        
        \item Determines set of all valid instances of a schema.
        
        \item Can only be compared, not quantitatively measured.
        \begin{description}
            \item[equivalence] \(\Equivalent{\SC{1}}{\SC{2}}\)
            \item[dominance] \(\Dominates{\SC{2}}{\SC{1}}\)
        \end{description}
        
        \item Mappings between schemas can preserve:
        \begin{description}
            \item[information] information capacity increases (\(\Rightarrow\Dominates{\SC{2}}{\SC{1}}\))
            \item[equivalence] information capacity stays the same (\(\Rightarrow\Equivalent{\SC{1}}{\SC{2}}\))
        \end{description}
    
    \end{itemize}
    
    \flushright\tiny
    Hull (1986) \emph{Relative information capacity of simple relational database schemata} \\
    Miller (1993) \emph{The use of information capacity in schema integration and translation}
}


%%%%%%%%%%%%%%%%%%%%%%%%%%%%%%%%%%%%%%%%%%%%%%%%%%%%%%%%%%%%%%%%%%%%%%%%%%%%%%%%


\frame
{
    \frametitle{Schema intension graphs (SIGs)}
    
    \structurebf{Nodes = typed domains (sets of values)}
    \begin{itemize}
        \item simple
        \item constructed (\(\times\), \(+\))
    \end{itemize}
    \medskip
    
    \structurebf{Edges = associations between domains} {\tiny (binary relations)}
    \begin{itemize}
        \item Simple constraints:
        \begin{itemize}
            \item total (\(A \Total B\)): defined for all \(a \in A\)
            \item surjective (\(A \Surjective B\)): defined for all \(b \in B\)
            \item functional (\(A \Functional B\)): \(a\) maps to at most one \(b\)
            \item injective (\(A \Injective B\)): \(b\) maps to at most one \(a\)
            \item bijective (\(A \Bijective B\)): all four
        \end{itemize}
        \item Selection (\(A \SelectionEdge B\)): \(B\) is a subset of \(A\). {\tiny (\(\Rightarrow B\) is a specialisation or subtype of \(A\))}
        \item Projection (\(A \ProjectionEdge B\)): \(B\) is a projection of \(A\). {\tiny (\(\Rightarrow A\) is a constructed tuple type)}
    \end{itemize}

    \flushright\tiny
    Miller (1994) \emph{Schema equivalence in heterogeneous systems: Bridging theory and practice} \\
    Miller (1994) \emph{Schema intension graphs: A formal model for the study of schema equivalence}
}


%%%%%%%%%%%%%%%%%%%%%%%%%%%%%%%%%%%%%%%%%%%%%%%%%%%%%%%%%%%%%%%%%%%%%%%%%%%%%%%%


\frame<1-2>[label=frame.SIG.example]
{
    \frametitle{Example SIG for relvar \(\bm{S\{\Sno\text{,}\, \Sname\text{,}\, \Status\text{,}\, \City\}}\)}
    \framesubtitle{\uncover<3>{\hyperlink{frame.SIG.Sone}{\beamerreturnbutton{Back}}}}
    
    \centering\Large
    \begin{tikzpicture}[level distance=18mm]
        \node (TS) {\(\Type{S}\)};
        \node (TTS) [right=1.5cm of TS] {\(\TT{S}\)}
            [sibling distance=4.5cm]
            child[arrows={<{crossbar}-{crossbar}>}] {node (TSno) {\(\Type{\Sno}\)}}
            child[arrows={:{crossbar}-{crossbar}>}] {node (TSSC) {\(\TSSC\)}
                [sibling distance=22mm]
                child[arrows={:{crossbar}-{crossbar}>}] {node (TSname) {\(\Type{\Sname}\)}}
                child[arrows={:{crossbar}-{crossbar}>}] {node (TStatus) {\(\Type{\Status}\)}}
                child[arrows={:{crossbar}-{crossbar}>}] {node (TCity) {\(\Type{\City}\)}}};
        
        \node (TTSexpansion) [right=-2mm of TTS,visible on=<2>,alert] {\tiny\(= \Type{\Sno} \times \Type{\Sname} \times \Type{\Status} \times \Type{\City}\)};
        
        % TS to T{S}
        \draw[arrows={:{crossbar}-{crossbar}:}] (TS) to (TTS);
        
        % FD
        \draw[arrows={:{crossbar}-{crossbar}>}] (TSno) to node[below=-2pt] {\scriptsize\emph{key}} (TSSC);
        
        % projection edges
        \ProjectionAnnotation[above left=-2pt and -4pt]{TTS}{TSno}
        \ProjectionAnnotation[above right=-2pt and -4pt]{TTS}{TSSC}

        \ProjectionAnnotation[above left=-2pt and -4pt]{TSSC}{TSname}
        \ProjectionAnnotation[right=-2pt]{TSSC}{TStatus}
        \ProjectionAnnotation[above right=-2pt and -4pt]{TSSC}{TCity}
    \end{tikzpicture}
}


%%%%%%%%%%%%%%%%%%%%%%%%%%%%%%%%%%%%%%%%%%%%%%%%%%%%%%%%%%%%%%%%%%%%%%%%%%%%%%%%


\begin{frame}[fragile]
    \frametitle{SIGs for relational schemas}
    
    \centering
    \begin{tabular}{lll}
        \toprule
        \textbf{Relational} &   \textbf{SIG}                    &   \textbf{Notation}   \\
        \midrule
        attribute type      &   simple domain                   &   \(\Type{A}\)    \\
        tuple type          &   constructed domain (\(\times\)) &   \(\TT{R}\) or \(\Type{A} \times \Type{B} \times \dotsc\)  \\
        relation type       &   simple domain\footnote{Or possibly a complex constructed type involving union?}               &   \(\Type{R}\)    \\
        \midrule
        key\footnote{Implied by functional dependency.}       &   func., total, surj.\ edge                 &   \(\Type{A} \KeyEdge \Type{B}\)    \\
        foreign key\footnote{Inclusion dependency.}         &   —                               &   —   \\
        \midrule
        restrict (\(\RelRestrict\))  &   selection edge  &   \(A \SelectionEdge B\) or \(A \TrivialSelection B\)\footnote{Trivial case.} \\
        project (\(\RelProject\))  &   projection edge  &   \(A \ProjectionEdge B\) or \(A \TrivialProjection B\) \\
        \bottomrule
    \end{tabular}
\end{frame}


%%%%%%%%%%%%%%%%%%%%%%%%%%%%%%%%%%%%%%%%%%%%%%%%%%%%%%%%%%%%%%%%%%%%%%%%%%%%%%%%


\frame
{
    \frametitle{Comparing and transforming SIGs}
    
    \begin{itemize}
    
        \item Isomorphism between SIGs for \(\SC{1}\) and \(\SC{2}\) (both structure and semantics) indicates information equivalence:
        \begin{itemize}
            \item \(\Equivalent{\SC{1}}{\SC{2}}\) if we can make \(\SC{1}\) isomorphic to \(\SC{2}\) using equivalence preserving transformations {\tiny (or vice versa)}
            \item \(\Dominates{\SC{2}}{\SC{1}}\) if we can make \(\SC{1}\) isomorphic to \(\SC{2}\) using information preserving transformations {\tiny (or vice versa)}
            \item \(\SC{1}\) \& \(\SC{2}\) non-comparable if we can’t make them isomorphic
        \end{itemize}
        \medskip
        
        \item Transforming a SIG effectively transforms the schema.
        \begin{itemize}
            \item delete annotation, \(\RelRestrict\), \(\RelProject\) (\(\preceq\))
            \item move and/or copy edge (\(\preceq\))
            \item create/delete duplicate node: \(A \Rightarrow A \TrivialSelection A'\) (\(\equiv\))
            \item delete bijective (trivial) selection edge: \(A \TrivialSelection A' \Rightarrow A \quad A'\) (\(\preceq\))
            \item create recursive bijective selection edge: \(A \Rightarrow\)\smash{\tikz[baseline={(A.base)},every loop/.style={out=-25,in=25,bijection}]{\node (A) {\(A\)}; \path (A) edge [loop right] node {\scriptsize\(\RelRestrict\)} ();}}(\(\equiv\))
        \end{itemize}
        
    \end{itemize}
}


%%%%%%%%%%%%%%%%%%%%%%%%%%%%%%%%%%%%%%%%%%%%%%%%%%%%%%%%%%%%%%%%%%%%%%%%%%%%%%%%


\frame
{
    \frametitle{Gotta catch ’em all…}
    \framesubtitle{(constraints, that is)}
    
    \begin{columns}
        \begin{column}[T]{0.5\textwidth}
            \structurebf{Consider}
            
            \small
            \begin{itemize}
                \item \(S\{\Sno\text{,}\,\Sname\text{,}\,\Status\text{,}\,\City\}\)
                \item \(\LS = \RelRestrict_{\City = \text{`\emph{London}'}}S\)
                \item \(\NLS = \RelRestrict_{\City \ne \text{`\emph{London}'}}S\)
                \item Base schema \(\SC{0} = \{S\text{,}\, \LS\text{,}\, \NLS\}\)
                \item Sub-schemas \(\SC{1} = \{S\}\), \(\SC{2} = \{\LS\text{,}\, \NLS\}\), \(\SC{3} = \{LS\}\)
                \item Date says \((\SC{1}\text{,}\, \SC{2})\) are informa- tion equivalent, \((\SC{1}\text{,}\, \SC{3})\) aren’t.\tinyskip
                {\tiny (due to information hiding)}
            \end{itemize}
            
            \alert{To construct truly comparable SIGs, we need to propagate all the constraints that we can from \(\SC{0}\) to \(\SC{1}\), \(\SC{2}\), and \(\SC{3}\).}
        \end{column}
        \begin{column}[T]{0.5\textwidth}
            \begin{uncoverenv}<2->
                \structurebf{Constraints in \(\bm{\SC{0}}\)}
            
                \tiny
                \begin{ConstraintList}
                    \item \(\{\Sno\} \rightarrow \{\Sname\text{,}\, \Status\text{,}\, \City\}\) holds in (a) \(S\); (b) \(\LS\); and (c) \(\NLS\) [key]
    
                    \item \(\RelProject_{\{\Sno\}}\LS \subseteq \RelProject_{\{\Sno\}}S\) [FK \(\LS \rightarrow S\)]
    
                    \item \(\RelProject_{\{\Sno\}}\NLS \subseteq \RelProject_{\{\Sno\}}S\) [FK \(\NLS \rightarrow S\)]
    
                    \item \(S = \LS \RelUnion \NLS\)
    
                    \item \(\RelRestrict_{\City \ne \text{`\emph{London}'}}\LS = \emptyset\)
    
                    \item \(\RelRestrict_{\City = \text{`\emph{London}'}}\NLS = \emptyset\)
    
                    \item \(\LS \RelIntersect \NLS = \emptyset\)
    
                    \item (a) \(\Type{S} = \Type{\LS} \RelUnion \Type{\NLS}\); (b) \(\TT{S} = \TT{\LS} \RelUnion \TT{\NLS}\)
    
                    \item (a) \(\Type{\LS} \subset \Type{S}\); (b) \(\TT{\LS} \subset \TT{S}\)
    
                    \item (a) \(\Type{\NLS} \subset \Type{S}\); (b) \(\TT{\NLS} \subset \TT{S}\)
    
                    \item \(\CityLondon \subset \Type{\City}\)
    
                    \item \((\TCityMinusLondon) \subset \Type{\City}\)
    
                    \item \(\TSSL \subset \TSSC\)
    
                    \item \(\TSSNL \subset \TSSC\)
                \end{ConstraintList}
            \end{uncoverenv}
        \end{column}
    \end{columns}
}


%%%%%%%%%%%%%%%%%%%%%%%%%%%%%%%%%%%%%%%%%%%%%%%%%%%%%%%%%%%%%%%%%%%%%%%%%%%%%%%%


\frame
{
    \frametitle{Propagating constraints to sub-schemas}
    \framesubtitle{(ignoring \(\bm{\text{C}_{2}}\) and \(\bm{\text{C}_{3}}\) for now)}
    
    \structurebf{Directly propagate constraints that reference no relvars}
    \begin{itemize}
        \item[] \(\text{C}_{11}\), \(\text{C}_{12}\), \(\text{C}_{13}\), \(\text{C}_{14}\)
    \end{itemize}
    \medskip
    
    \structurebf{Directly propagate constraints that reference sub-schema relvars only}
    \begin{itemize}
        \item[\(\SC{1}\)] \(\text{C}_{1}\)(a)
        \item[\(\SC{2}\)] \(\text{C}_{1}\)(b), \(\text{C}_{1}\)(c), \(\text{C}_{5}\), \(\text{C}_{6}\), \(\text{C}_{7}\)
        \item[\(\SC{3}\)] \(\text{C}_{1}\)(b), \(\text{C}_{5}\)
    \end{itemize}
    \medskip
    
    \structurebf{Rewrite remaining constraints in terms of sub-schema relvars {\tiny (if possible)}}
    
    \begin{itemize}
        \item[\(\SC{1}\)] Replace \(\LS\) with \(\RelRestrict_{\City = \text{`\emph{London}'}}\,S\) and \(\NLS\) with \(\RelRestrict_{\City \ne \text{`\emph{London}'}}\,S\) in \(\text{C}_{1}\)(b), \(\text{C}_{1}\)(c), \(\text{C}_{4}\), \(\text{C}_{5}\), \(\text{C}_{6}\), \(\text{C}_{7}\), \(\text{C}_{8}\), \(\text{C}_{9}\), \(\text{C}_{10}\).
        
        \item[\(\SC{2}\)] Replace \(S\) with \(\LS \RelUnion \NLS\) in \(\text{C}_{1}\)(a), \(\text{C}_{4}\), \(\text{C}_{8}\), \(\text{C}_{9}\), \(\text{C}_{10}\).
        
        \item[\(\SC{3}\)] Can’t rewrite anything.
    \end{itemize}
}


%%%%%%%%%%%%%%%%%%%%%%%%%%%%%%%%%%%%%%%%%%%%%%%%%%%%%%%%%%%%%%%%%%%%%%%%%%%%%%%%


\frame
{
    \frametitle{Propagating constraints to sub-schemas}
    
    \structurebf{Example}
    \medskip
    
    \begin{tabular}{lll}
        \textcolor{structure}{\(\SC{0}\)}    &   \(\text{C}_{4}\)            &   \(S = \LS \RelUnion \NLS\)    \\[4pt]
                                        &                               &   \Large\quad\(\Downarrow\)  \\[4pt]
        \textcolor{structure}{\(\SC{1}\)}    &   \(\text{C}_{4}^{\SC{1}}\)   &   \(S = (\RelRestrict_{\City = \text{`\emph{London}'}}\,S) \RelUnion (\RelRestrict_{\City \ne \text{`\emph{London}'}}\,S)\)    \\[8pt]
        \textcolor{structure}{\(\SC{2}\)}    &   \(\text{C}_{4}^{\SC{2}}\)   &   \(\LS \RelUnion \NLS = \LS \RelUnion \NLS\) \\[8pt]
        \textcolor{structure}{\(\SC{3}\)}    &  —                            &   (can’t be propagated)   \\
    \end{tabular}
}


%%%%%%%%%%%%%%%%%%%%%%%%%%%%%%%%%%%%%%%%%%%%%%%%%%%%%%%%%%%%%%%%%%%%%%%%%%%%%%%%


\frame<1>[label=frame.SIG.Sone]
{
    \frametitle{The SIGs: \(\bm{\SC{1} = \{S\}}\)}
    \framesubtitle{\alt<1>{\hyperlink{frame.SIG.example<3>}{\beamergotobutton{Compare}}}{\hyperlink{frame.SIG.transform<9>}{\beamerreturnbutton{Back}}}}
    
    \centering
    \begin{tikzpicture}[ampersand replacement=\&,transform canvas={scale=0.8}]
        \matrix[row sep=0.5cm]
        { 
            \node (TLS) {\(\Type{\LSsub}\)};    \&[7mm]  \node (TTLS) {\(\TT{\LSsub}\)};     \&[7mm]                              \&[4mm]                                  \&[-4mm] \node (TSSL) {\(\StackTSSL\)};      \&                                       \&   \node (LSCity) {\(\CityLondon\)};   \\
                                                \&                                           \&                                   \&   \node (TSname) {\(\Type{\Sname}\)};    \\
            \node (TLSPlusNLS) {\(\Type{S}\)};  \&       \node (TTLSPlusNLS) {\(\TT{S}\)};   \&   \node (TSno) {\(\Type{\Sno}\)}; \&                                       \&       \node (TSSC) {\(\StackTSSC\)};      \&                                       \&   \node (TCity) {\(\Type{City}\)};   \\
                                                \&                                           \&                                   \&                                       \&                                           \&   \node (TStatus) {\(\Type{\Status}\)};  \\
            \node (TNLS) {\(\Type{\NLSsub}\)};  \&       \node (TTNLS) {\(\TT{\NLSsub}\)};   \&                                   \&                                       \&       \node (TSSNL) {\(\StackTSSNL\)};    \&                                       \&   \node (NLSCity) {\(\StackTCityMinusLondon\)}; \\
        };
        
        \graph{
            % relation types to tuple types
            { [edges=surtotal]
                { (TLS), (TNLS), (TLSPlusNLS) } -- { (TTLS), (TTNLS), (TTLSPlusNLS) },
            };
            
            % FDs
            (TSno) -- [funcdep left,bend left] (TSSL);
            (TSno) -- [funcdep right] (TSSNL);
            
            % projection edges
            { [edges=bijection,edge label={\scriptsize\(\RelProject\)}]
                { (TTLSPlusNLS), (TTLS) } -- (TSno)
            };
            
            { [edges=bijection,edge label'={\scriptsize\(\RelProject\)}]
                (TTNLS) -- (TSno)
            };
            
            { [edges=projection left]
                (TTLS)  -- (TSSL),
                (TSSL)  -- { (TStatus), (LSCity) },
            };
            
            { [edges=projection right]
                (TTNLS) -- (TSSNL),
                (TSSC)  -- { (TSname),  (TStatus) },
                (TSSL)  -- (TSname),
                (TSSNL) -- { (TStatus), (NLSCity), (TSname) },
            };
            
            % selection edges
            { [edges=selection left]
                { (TLSPlusNLS), (TTLSPlusNLS) } -- { (TLS), (TTLS) },
                (TSSC)  -- { (TSSL), (TSSNL) },
                (TCity) -- (NLSCity),
            };
            { [edges=selection right]
                { (TLSPlusNLS), (TTLSPlusNLS) } -- { (TNLS), (TTNLS) },
                (TCity) -- (LSCity),
            };
            
            % insert "gaps" into crossed edges
            { [edges={white,line width=1.5mm}]
                (TSno) -- (TSSC),
                (TSSC) -- (TCity),
                (TTLSPlusNLS) -- [bend right] (TSSC),
            };
            (TSno) -- [funcdep right] (TSSC);
            (TSSC) -- [projection left] (TCity);
            (TTLSPlusNLS) -- [projection right,bend right] (TSSC),
        };
        
        \node[below=5mm of NLSCity,alert] (caption) {\scriptsize\shortstack[r]{\textbf{Note:} \(\LSsub = \RelRestrict_{\City = \text{`\emph{London}'}}\,S\) \\ \(\NLSsub = \RelRestrict_{\City \ne \text{`\emph{London}'}}\,S\)}};
    \end{tikzpicture}
}


%%%%%%%%%%%%%%%%%%%%%%%%%%%%%%%%%%%%%%%%%%%%%%%%%%%%%%%%%%%%%%%%%%%%%%%%%%%%%%%%


\frame
{
    \frametitle{The SIGs: \(\bm{\SC{2} = \{\LS\text{,}\, \NLS\}}\)}
    \framesubtitle{\invisible{\beamergotobutton{Compare}}}
    
    \centering
    \begin{tikzpicture}[ampersand replacement=\&,transform canvas={scale=0.8}]
        \matrix[row sep=0.5cm]
        {
            \node (TLS) {\(\Type{\LS}\)};                  \&[7mm]  \node (TTLS) {\(\TT{\LS}\)};                    \&[7mm]                              \&[4mm]                                  \&[-4mm] \node (TSSL) {\(\StackTSSL\)};      \&                                       \&   \node (LSCity) {\(\CityLondon\)};   \\
                                                        \&                                                       \&                                   \&   \node (TSname) {\(\Type{\Sname}\)};    \\
            \node (TLSPlusNLS) {\(\StackTLSPlusNLS\)};  \&       \node (TTLSPlusNLS) {\(\StackTTLSPlusNLS\)};    \&   \node (TSno) {\(\Type{\Sno}\)};    \&                                       \&       \node (TSSC) {\(\StackTSSC\)};      \&                                       \&   \node (TCity) {\(\Type{City}\)};   \\
                                                        \&                                                       \&                                   \&                                       \&                                           \&   \node (TStatus) {\(\Type{\Status}\)};  \\
            \node (TNLS) {\(\Type{\NLS}\)};                \&       \node (TTNLS) {\(\TT{\NLS}\)};                  \&                                   \&                                       \&       \node (TSSNL) {\(\StackTSSNL\)};    \&                                       \&   \node (NLSCity) {\(\StackTCityMinusLondon\)}; \\
        };
        
        \graph{
            % relation types to tuple types
            { [edges=surtotal]
                { (TLS), (TNLS), (TLSPlusNLS) } -- { (TTLS), (TTNLS), (TTLSPlusNLS) },
            };
            
            % FDs
            (TSno) -- [funcdep left,bend left] (TSSL);
            (TSno) -- [funcdep right] (TSSNL);
            
            % projection edges
            { [edges=bijection,edge label={\scriptsize\(\RelProject\)}]
                { (TTLSPlusNLS), (TTLS) } -- (TSno)
            };
            
            { [edges=bijection,edge label'={\scriptsize\(\RelProject\)}]
                (TTNLS) -- (TSno)
            };
            
            { [edges=projection left]
                (TTLS)  -- (TSSL),
                (TSSL)  -- { (TStatus), (LSCity) },
            };
            
            { [edges=projection right]
                (TTNLS) -- (TSSNL),
                (TSSC)  -- { (TSname),  (TStatus) },
                (TSSL)  -- (TSname),
                (TSSNL) -- { (TStatus), (NLSCity), (TSname) },
            };
            
            % selection edges
            { [edges=selection left]
                { (TLSPlusNLS), (TTLSPlusNLS) } -- { (TLS), (TTLS) },
                (TSSC)  -- { (TSSL), (TSSNL) },
                (TCity) -- (NLSCity),
            };
            { [edges=selection right]
                { (TLSPlusNLS), (TTLSPlusNLS) } -- { (TNLS), (TTNLS) },
                (TCity) -- (LSCity),
            };
            
            % insert "gaps" into crossed edges
            { [edges={white,line width=1.5mm}]
                (TSno) -- (TSSC),
                (TSSC) -- (TCity),
                (TTLSPlusNLS) -- [bend right] (TSSC),
            };
            (TSno) -- [funcdep right] (TSSC);
            (TSSC) -- [projection left] (TCity);
            (TTLSPlusNLS) -- [projection right,bend right] (TSSC),
        };
        
        \node[below=4cm of current page.south west,structure] (caption) {\Large This is already isomorphic to the SIG for \(\SC{1}\), so \(\Equivalent{\SC{1}}{\SC{2}}\).};
    \end{tikzpicture}
}


%%%%%%%%%%%%%%%%%%%%%%%%%%%%%%%%%%%%%%%%%%%%%%%%%%%%%%%%%%%%%%%%%%%%%%%%%%%%%%%%


\frame
{
    \frametitle{The SIGs: \(\bm{\SC{3} = \{\LS\}}\)}
    \framesubtitle{\invisible{\beamergotobutton{Compare}}}
    
    \centering
    \begin{tikzpicture}[ampersand replacement=\&,transform canvas={scale=0.8}]
        \matrix[row sep=0.5cm]
        { 
            \node (TLS) {\(\Type{\LS}\)};      \&[7mm]  \node (TTLS) {\(\TT{\LS}\)};        \&[7mm]                              \&[4mm]                                  \&[-4mm] \node (TSSL) {\(\StackTSSL\)};      \&                                       \&   \node (LSCity) {\(\CityLondon\)};   \\
                                            \&                                           \&                                   \&   \node (TSname) {\(\Type{\Sname}\)};    \\
            \node[invisible] (TLSPlusNLS) {\(\Type{S}\)}; \&       \node[invisible] (TTLSPlusNLS) {\(\TT{S}\)};   \&   \node (TSno) {\(\Type{\Sno}\)};    \&                                       \&       \node (TSSC) {\(\StackTSSC\)};      \&                                       \&   \node (TCity) {\(\Type{City}\)};   \\
                                            \&                                           \&                                   \&                                       \&                                           \&   \node (TStatus) {\(\Type{\Status}\)};  \\
            \node[invisible] (TNLS) {\(\Type{\NLS}\)};    \&       \node[invisible] (TTNLS) {\(\TT{\NLS}\)};      \&                                   \&                                       \&       \node (TSSNL) {\(\StackTSSNL\)};    \&                                       \&   \node (NLSCity) {\(\StackTCityMinusLondon\)}; \\
        };
        
        \graph{
            % relation types to tuple types
            { [edges=surtotal]
                (TLS)        -- (TTLS),
                (TLSPlusNLS) --[invisible] (TTLSPlusNLS),
                (TNLS)       --[invisible] (TTNLS),
            };
            
            % FDs
            (TSno) -- [funcdep left,bend left,invisible] (TSSL);
            (TSno) -- [funcdep right,invisible] (TSSNL);
            
            % projection edges
            { [edges=bijection,edge label={\scriptsize\(\RelProject\)}]
                (TTLS) -- (TSno)
            };
            
            { [edges=projection left]
                (TTLS)  -- (TSSL),
                (TSSL)  -- { (TStatus), (LSCity) },
                (TTLSPlusNLS) --[invisible] (TSno),
            };
            
            { [edges=projection right]
                (TSSC)  -- { (TSname),  (TStatus) },
                (TSSL)  -- (TSname),
                (TSSNL) -- { (TStatus), (NLSCity), (TSname) },
                (TTNLS)  --[invisible] (TSno),
                (TTNLS)  -- [arrows={:->},invisible] (TSSNL),
            };
            
            % selection edges
            { [edges=selection left]
                (TSSC)  -- { (TSSL), (TSSNL) },
                (TCity) -- (NLSCity),
            };
            { [edges=selection right]
                (TCity) -- (LSCity),
            };
            
            { [edges=bijection,edge label={\scriptsize\(\RelRestrict\)}]
                { (TLS), (TTLS) }               -- [invisible] { (TLSPlusNLS), (TTLSPlusNLS) },
                { (TLSPlusNLS), (TTLSPlusNLS) } -- [invisible] { (TNLS), (TTNLS) },
            };
            
            { [edges=selection right]
                { (TLSPlusNLS), (TTLSPlusNLS) } --[invisible] { (TLS), (TTLS) },
            };
            
            { [edges=selection left]
                { (TLSPlusNLS), (TTLSPlusNLS) } --[invisible] { (TNLS), (TTNLS) },
            };
            
            % insert "gaps" into crossed edges
            { [edges={white,line width=1.5mm}]
                (TSno) -- (TSSC),
                (TSSC) -- (TCity),
                (TTLSPlusNLS) -- [bend right,invisible] (TSSC),
            };
            (TSno) -- [funcdep right] (TSSC);
            (TSSC) -- [projection left] (TCity);
            (TTLSPlusNLS) -- [projection right,arrows={:{crossbar}->},bend right,invisible] (TSSC);
        };
        
        \node[below=4cm of current page.south west,structure] (caption) {\Large Similarities with \(\SC{1}\) but not identical, so we need to transform…};
    \end{tikzpicture}
}


%%%%%%%%%%%%%%%%%%%%%%%%%%%%%%%%%%%%%%%%%%%%%%%%%%%%%%%%%%%%%%%%%%%%%%%%%%%%%%%%


\frame[label=frame.SIG.transform]
{
    \frametitle{Transforming \(\bm{\SC{3} \Rightarrow \SC{1}}\)}
    \framesubtitle{\alt<1-8>{\textcolor{structure}{blue} = input to transformation, \textcolor{alert}{red} = output\invisible{\beamergotobutton{Compare}}}{\hyperlink{frame.SIG.Sone<2>}{\beamergotobutton{Compare}}}}
    
    \centering
    \begin{tikzpicture}[ampersand replacement=\&,transform canvas={scale=0.8}]
        \matrix[row sep=0.5cm]
        { 
            \node[input on=<2>] (TLS) {\(\Type{\LS}\)};      \&[7mm]  \node[input on=<2>] (TTLS) {\(\TT{\LS}\)};        \&[7mm]                              \&[4mm]                                  \&[-4mm] \node (TSSL) {\(\StackTSSL\)};      \&                                       \&   \node (LSCity) {\(\CityLondon\)};   \\
                                            \&                                           \&                                   \&   \node (TSname) {\(\Type{\Sname}\)};    \\
            \node[visible on=<2->,output on=<2>,input on=<3>] (TLSPlusNLS) {\(\Type{S}\)}; \&       \node[visible on=<2->,output on=<2>,input on=<3>] (TTLSPlusNLS) {\(\TT{S}\)};   \&   \node (TSno) {\(\Type{\Sno}\)};    \&                                       \&       \node (TSSC) {\(\StackTSSC\)};      \&                                       \&   \node (TCity) {\(\Type{City}\)};   \\
                                            \&                                           \&                                   \&                                       \&                                           \&   \node (TStatus) {\(\Type{\Status}\)};  \\
            \node[visible on=<3->,output on=<3>] (TNLS) {\(\Type{\NLS}\)};    \&       \node[visible on=<3->,output on=<3>] (TTNLS) {\(\TT{\NLS}\)};      \&                                   \&                                       \&       \node (TSSNL) {\(\StackTSSNL\)};    \&                                       \&   \node (NLSCity) {\(\StackTCityMinusLondon\)}; \\
        };
        
        \graph{
            % relation types to tuple types
            { [edges=surtotal]
                (TLS)        --[input on=<4>] (TTLS),
                (TLSPlusNLS) --[visible on=<4->,output on=<4>] (TTLSPlusNLS),
                (TNLS)       --[visible on=<4->,output on=<4>] (TTNLS),
            };
            
            % FDs
            (TSno) -- [funcdep left,bend left,visible on=<7->,output on=<7>] (TSSL);
            (TSno) -- [funcdep right,visible on=<7->,output on=<7>] (TSSNL);
            
            % projection edges
            { [edges=bijection,edge label={\scriptsize\(\RelProject\)}]
                (TTLS)  --[input on=<5>] (TSno),
                (TTLSPlusNLS) --[visible on=<5->,output on=<5>] (TSno)
            };
            
            { [edges=bijection,edge label'={\scriptsize\(\RelProject\)}]
                (TTNLS)  --[visible on=<5->,output on=<5>] (TSno)
            };
            
            { [edges=projection left]
                (TTLS)  --[input on=<6>] (TSSL),
                (TSSL)  -- { (TStatus), (LSCity) }
            };
            
            { [edges=projection right]
                (TTNLS)  -- [arrows={:->},visible on=<6->,output on=<6>] (TSSNL),
                (TSSC)  -- { (TSname),  (TStatus) },
                (TSSL)  -- (TSname),
                (TSSNL) -- { (TStatus), (NLSCity), (TSname) }
            };

            % selection edges
            { [edges=selection left]
                (TSSC)  -- { (TSSL), (TSSNL) },
                (TCity) -- (NLSCity),
            };
            { [edges=selection right]
                (TCity) -- (LSCity),
            };
            
            { [edges=bijection,edge label'={\scriptsize\(\RelRestrict\)}]
                { (TLS), (TTLS) }               -- [visible on=<2-7>,output on=<2>] { (TLSPlusNLS), (TTLSPlusNLS) },
                { (TLSPlusNLS), (TTLSPlusNLS) } -- [visible on=<3-7>,output on=<3>] { (TNLS), (TTNLS) },
            };
            
            { [edges=selection left]
                { (TLSPlusNLS), (TTLSPlusNLS) } --[visible on=<8->] { (TLS), (TTLS) },
            };
            
            { [edges=selection right]
                { (TLSPlusNLS), (TTLSPlusNLS) } --[visible on=<8->] { (TNLS), (TTNLS) },
            };
            
            % insert "gaps" into crossed edges
            { [edges={white,line width=1.5mm}]
                (TSno) -- (TSSC),
                (TSSC) -- (TCity),
                (TTLSPlusNLS) -- [bend right,visible on=<6->] (TSSC),
            };
            (TSno) -- [funcdep right,input on=<7>] (TSSC);
            (TSSC) -- [projection left] (TCity);
            (TTLSPlusNLS) -- [projection right,arrows={:{crossbar}->},bend right,visible on=<6->,output on=<6>] (TSSC);
        };
        
        \begin{scope}[circle,visible on=<8>,output on=<8>]
            \node[draw] (c1) at (TLSPlusNLS) [above=11.5pt] {};
            \node[draw] (c2) at (TLSPlusNLS) [below=11.5pt] {};
            \node[draw] (c3) at (c1 -| TTLSPlusNLS) {};
            \node[draw] (c4) at (c2 -| TTLSPlusNLS) {};
        \end{scope}
        
        \node[below=4cm of current page.south west,structure] (caption) {
            \Large
            \only<1>{Initial state}
            \only<2>{Duplicate \(\Type{\LS}\) and \(\TT{\LS}\) [\(\equiv\)]}
            \only<3>{Duplicate \(\Type{S}\) and \(\TT{S}\) [\(\equiv\)]}
            \only<4>{Copy and move \(\Type{\LS} \SurTotal \TT{\LS}\) edge (\(\times 2\)) [\(\preceq\)]}
            \only<5>{Copy and move \(\TT{\LS} \ProjectionEdge \Type{\Sno}\) edge (\(\times 2\)) [\(\preceq\)]}
            \only<6>{Copy and move \(\TT{\LS} \ProjectionEdge \TSSL\) edge (\(\times 2\)) [\(\preceq\)]}
            \only<7>{Copy and move \(\Type{\Sno} \KeyEdge \TSSC\) edge (\(\times 2\)) [\(\preceq\)]}
            \only<8>{Remove totality annotations [\(\preceq\)]}
            \only<9>{Final state: \emph{not} isomorphic to SIG for \(\SC{1} \Rightarrow\) non-comparable.}
        };
    \end{tikzpicture}
}


%%%%%%%%%%%%%%%%%%%%%%%%%%%%%%%%%%%%%%%%%%%%%%%%%%%%%%%%%%%%%%%%%%%%%%%%%%%%%%%%


\frame
{
    \frametitle{So what does this mean?}
    
    \structurebf{\(\bm{\SC{1}}\) and \(\bm{\SC{2}}\) are information equivalent}
    
    \begin{itemize}
    
        \item[\(\Rightarrow\)] all view updates between them should be OK
    
    \end{itemize}
    \bigskip
    
    \structurebf{\(\bm{\SC{1}}\) and \(\bm{\SC{3}}\) aren’t information equivalent}
    
    \begin{itemize}
    
        \item[\(\Rightarrow\)] some view updates between them will cause problems
    
    \end{itemize}
    \bigskip
    
    \structurebf{This agrees with what Date found} {\tiny (yay!)}
}


%%%%%%%%%%%%%%%%%%%%%%%%%%%%%%%%%%%%%%%%%%%%%%%%%%%%%%%%%%%%%%%%%%%%%%%%%%%%%%%%


\frame
{
    \frametitle{Now what?}
    
    \begin{itemize}
    
        \item Is this a useful technique? {\tiny (try analysing some more complex examples)}
        
        \item Are inclusion dependencies important? \\
        If so, how to represent them?
        
        \item Do we need the whole transformation process? Determining whether or not all constraints can be propagated from the base schema may already be enough.
        
        \item Could it help provide a more formal basis for Date’s “pragma”?
        
        \item Might it reveal some interesting new insights about the view update problem?
        
        \item Submitted to APCCM 2017.
    
    \end{itemize}
}


%%%%%%%%%%%%%%%%%%%%%%%%%%%%%%%%%%%%%%%%%%%%%%%%%%%%%%%%%%%%%%%%%%%%%%%%%%%%%%%%


\frame
{
    \centering\Huge
    \structurebf{Questions?}
}


%%%%%%%%%%%%%%%%%%%%%%%%%%%%%%%%%%%%%%%%%%%%%%%%%%%%%%%%%%%%%%%%%%%%%%%%%%%%%%%%


\againframe<3>{frame.SIG.example}

\againframe<2>{frame.SIG.Sone}


%%%%%%%%%%%%%%%%%%%%%%%%%%%%%%%%%%%%%%%%%%%%%%%%%%%%%%%%%%%%%%%%%%%%%%%%%%%%%%%%


\end{document}
